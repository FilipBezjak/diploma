\documentclass[a4paper,12pt]{article}
\usepackage[slovene]{babel}
\usepackage[utf8]{inputenc}
\usepackage[T1]{fontenc}
\usepackage{lmodern}
\usepackage{verbatim}

\usepackage{url}
\usepackage{graphicx}
\usepackage{amsmath}
\usepackage{amsthm}
\usepackage{dsfont}
\usepackage{amssymb}
\usepackage{hyperref}



\makeatletter
\DeclareRobustCommand{\sqcdot}{\mathbin{\mathpalette\morphic@sqcdot\relax}}
\newcommand{\morphic@sqcdot}[2]{%
\sbox\z@{$\m@th#1\centerdot$}%
\ht\z@=.33333\ht\z@
\vcenter{\box\z@}%
}
\makeatother

\makeatletter
\DeclareRobustCommand{\k}{
    \mathcal{K}
}
\makeatletter
\DeclareRobustCommand{\si}{
    \bar{\sigma}
}

\newcommand\mymathop[1]{\mathop{\operatorname{#1}}}

\title{Minimalni končni modeli prostorov}
\author{Filip Bezjak \\ Mentor: dr. Petar Pavešić}


\setlength\parindent{24pt}

\theoremstyle{definition}
\newtheorem{definicija}{Definicija}

\theoremstyle{plain}
\newtheorem{izrek}{Izrek}

\theoremstyle{plain}
\newtheorem{trditev}{Trditev}

\theoremstyle{definition}
\newtheorem{primer}{Primer}

\theoremstyle{plain}
\newtheorem{opomba}{Opomba}

\theoremstyle{plain}
\newtheorem{lema}{Iema}

\newenvironment{dokaz}{\begin{proof}[\bfseries\upshape\proofname]}{\end{proof}}

\begin{document}


\section{Zanke v Hassejevem diagramu}

Pokazali bomo, kako se fundamentalna grupa končnega $T_0$ prostora izraža preko prirejenega Hassejevega diagrama.
Hassejev diagram končnega $T_0$ prostora $X$ označimo z $H(X)$, z $E(H(X))$ pa označimo množico njegovih robov.

\textit{Edge path} simplicialnega kompleksa $K$ je zaporedje
$(v_0,v_1)(v_1,v_2),...,(v_{r-1},v_{r})$ urejenih parov 
ogljišč, pri čemer je $\{v_1,v_{i+1}\}$ simpleks za vsak $i$. 
Če \textit{edge path} vsebuje dva zaporedna para 
$(v_i,v_{i+1})$ in $(v_{i+1},_v_{i+2})$ in je 
$\{v_i,v_{i+1},_v_{i+2}\}$ simpleks, potem ju lahko 
zamenjamo z parom $(v_i,v_{i+1})$ in dobimo ekvivalentno a 
krajšo pot. Za poti $(v_0,v_1)(v_1,v_2),...,(v_{r-1},v_{r})$ 
in $(u_0,u_1)(u_1,u_2),...,(u_{s-1},u_{s})$ definiramo stik 
poti.....???
Omejili se bomo na zanke, torej poti, ki se začnejo in končajo z $v_0$. Z $E(K,v_0)$ označimo množico ekvivalenčnih razredov zank z začetno točko $v_0$.....


Naj bo $(X,x_0)$ končen pointed $T_0$ prostor. Urejen par 
$e=(x,y)$ imenujemo $H-rob$ od $X$, če $(x,y)\in 
E(\mathcal{H}(\mathcal{X}))$, ali $(y,x)\in 
E(H(X))$. Točki $x$ rečem \textit{začetek} $x$ in označimo 
$x=\mathfrak{o}(e)$, točki $y$ pa \textit{konec} od $e$, 
pznačimo $\mathfrak{e}(e)=y$