\documentclass[a4paper,12pt]{article}
\usepackage[slovene]{babel}
\usepackage[utf8]{inputenc}
\usepackage[T1]{fontenc}
\usepackage{lmodern}
\usepackage{verbatim}

\usepackage{url}
\usepackage{graphicx}
\usepackage{amsmath}
\usepackage{amsthm}
\usepackage{dsfont}
\usepackage{amssymb}
\usepackage{hyperref}



\makeatletter
\DeclareRobustCommand{\sqcdot}{\mathbin{\mathpalette\morphic@sqcdot\relax}}
\newcommand{\morphic@sqcdot}[2]{%
\sbox\z@{$\m@th#1\centerdot$}%
\ht\z@=.33333\ht\z@
\vcenter{\box\z@}%
}
\makeatother

\makeatletter
\DeclareRobustCommand{\k}{
    \mathcal{K}
}



\makeatletter
\DeclareRobustCommand{\h}{
    \mathcal{H}
}
\makeatletter
\DeclareRobustCommand{\si}{
    \bar{\sigma}
}


\makeatletter
\DeclareRobustCommand{\pot}{
    $\h-$pot
}


\newcommand\mymathop[1]{\mathop{\operatorname{#1}}}

\title{Minimalni končni modeli prostorov}
\author{Filip Bezjak \\ Mentor: dr. Petar Pavešić}


\setlength\parindent{24pt}

\theoremstyle{definition}
\newtheorem{definicija}{Definicija}

\theoremstyle{plain}
\newtheorem{izrek}{Izrek}

\theoremstyle{plain}
\newtheorem{trditev}{Trditev}

\theoremstyle{definition}
\newtheorem{primer}{Primer}

\theoremstyle{plain}
\newtheorem{opomba}{Opomba}

\theoremstyle{plain}
\newtheorem{lema}{Iema}

\newenvironment{dokaz}{\begin{proof}[\bfseries\upshape\proofname]}{\end{proof}}

\begin{document}


\section{Zanke v Hassejevem diagramu}

Pokazali bomo, kako se fundamentalna grupa končnega $T_0$ 
prostora izraža preko prirejenega Hassejevega diagrama.
Hassejev diagram končnega $T_0$ prostora $X$ označimo z 
$\h(X)$, z $E(\h(X))$ pa označimo množico njegovih robov.

\textit{Edge path} simplicialnega kompleksa $K$ je zaporedje
$(v_0,v_1)(v_1,v_2),...,(v_{r-1},v_{r})$ urejenih parov 
ogljišč, pri čemer je $\{v_1,v_{i+1}\}$ simpleks za vsak $i$. 
Če \textit{edge path} vsebuje dva zaporedna para 
$(v_i,v_{i+1})$ in $(v_{i+1},_v_{i+2})$ in je 
$\{v_i,v_{i+1},_v_{i+2}\}$ simpleks, potem ju lahko 
zamenjamo z parom $(v_i,v_{i+1})$ in dobimo ekvivalentno a 
krajšo pot. Za poti $(v_0,v_1)(v_1,v_2),...,(v_{r-1},v_{r})$ 
in $(u_0,u_1)(u_1,u_2),...,(u_{s-1},u_{s})$ definiramo stik 
poti.....???
Omejili se bomo na zanke, torej poti, ki se začnejo in končaj
o z $v_0$. Z $E(K,v_0)$ označimo množico ekvivalenčnih razredov
 zank z začetno točko $v_0$.....


Naj bo $(X,x_0)$ končen pointed $T_0$ prostor. Urejen par 
$e=(x,y)$ imenujemo $\mathcal{H}-$rob od $X$, če $(x,y)\in 
E(\mathcal{H}(\mathcal{X}))$, ali $(y,x)\in 
E(\h(X))$. Točki $x$ rečem \textit{začetek} $x$ in označimo 
$x=\mathfrak{o}(e)$, točki $y$ pa \textit{konec} od $e$, 
označimo $\mathfrak{e}(e)=y$. \textit{Inverz} $\h-$roba $e=(x,y)$ je $\h-$rob $e^{-1}=(y,x)$

$\h-$pot v $(X,x_0)$ je zaporedje (lahko tudi prazno), $\h-$robov $\xi=e_1e_2\cdots e_n$, 
za katero velja, da je $\mathfrak{e}(e_i)=\mathfrak{o}(e_i+1)$, za vsak $0\leq i \leq n-1$.
 Začetek $\h-$poti $\xi$ je  $\mathfrak{o}(\xi)=e_1$, konez pa $\mathfrak{e}(\xi)=e_n$, 
 začetek in konec prazne poti je $\mathfrak{o}(\emptyset)=\mathfrak{e}(\emptyset)=x_0$
 Če je $\xi=e_1,e_2\cdots e_n$ $\h-$pot, definiramo $\overline{\xi}=e_n^{-1},\cdots 
 e_2^{-1}e_n^{-1}$. Če sta $\xi$ in $\xi'$ $\h-$poti in velja $\mathfrak{e}(\xi)=
 \mathfrak{e}(\xi')$, lahko definiramo produktno \pot $\xi\xi'$, kot zaporednje 
 $\h-$robov v $\xi$, ki mu sledi zaporednje $\h-$robov v $\xi'$.

 Za \pot $\xi=e_1e_2,\cdots e_n$ pravimo, da je \textit{monotona}, če je $e_i\in 
 E(\h(X))$ za vsak $1\leq i \leq n$ ali pa je $e_i^{-1}\in E(\h(X))$ za vsak $1\leq i \leq n$.
 \textit{Zanka} iz $x_0$ je \pot, ki se začne in konča v $x_0$. Za zanki $\xi$ in
  $\xi'$ rečemo, da sta blizu, če obstajajo monotone $\h-$poti $\xi_1,\xi_2,\xi_3,\xi_4$,
   take, da  sta množici $\{\xi,\xi'}$ in $\{\xi_1\xi_2\xi_3\xi_4,\xi_1\xi_4\}$ enaki.

 \begin{primer}
    poti ki sta si blizu.
 \end{primer}



\end{document}