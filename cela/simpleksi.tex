\documentclass[a4paper,12pt]{article}
\usepackage[slovene]{babel}
\usepackage[utf8]{inputenc}
\usepackage[T1]{fontenc}
\usepackage{lmodern}
\usepackage{verbatim}

\usepackage{url}
\usepackage{graphicx}
\usepackage{amsmath}
\usepackage{amsthm}
\usepackage{dsfont}
\usepackage{amssymb}
\usepackage{hyperref}



\makeatletter
\DeclareRobustCommand{\sqcdot}{\mathbin{\mathpalette\morphic@sqcdot\relax}}
\newcommand{\morphic@sqcdot}[2]{%
\sbox\z@{$\m@th#1\centerdot$}%
\ht\z@=.33333\ht\z@
\vcenter{\box\z@}%
}
\makeatother

\makeatletter
\DeclareRobustCommand{\k}{
    \mathcal{K}
}
\makeatletter
\DeclareRobustCommand{\si}{
    \bar{\sigma}
}

\newcommand\mymathop[1]{\mathop{\operatorname{#1}}}

\title{Minimalni končni modeli prostorov}
\author{Filip Bezjak \\ Mentor: dr. Petar Pavešić}


\setlength\parindent{24pt}

\theoremstyle{definition}
\newtheorem{definicija}{Definicija}

\theoremstyle{plain}
\newtheorem{izrek}{Izrek}

\theoremstyle{plain}
\newtheorem{trditev}{Trditev}

\theoremstyle{definition}
\newtheorem{primer}{Primer}

\theoremstyle{plain}
\newtheorem{opomba}{Opomba}

\theoremstyle{plain}
\newtheorem{lema}{Iema}

\newenvironment{dokaz}{\begin{proof}[\bfseries\upshape\proofname]}{\end{proof}}

\begin{document}

\maketitle


\section{Simpleksi}

    \textit{Simpleks} ali $n$-simpleks je $n$-razsežni analog trikotnika. Točka je $0$-simpleks, $1$-simpleks je daljica, $2$-simpleks je trikotnik,
$3$-simpleks je tetraeder. $n$-simpleks definiramo kot množico svojih $n+1$ oglišč. \textit{Simplicialni kompleks $K$} je sestavljen iz množice oglišč $V_K$ in množice simpleksov $S_K$, sestavljene iz končnih nepraznih podmnožic od $V_k$, pri čemer je vsak element $S_k$ simpleks in vsaka podmnožica simpleksa je simpleks. Pišemo $\sigma \in K$ in $v \in K$, če je $\sigma \in S_K$ ter $v \in V_K$. Dimenzija $K$ je enaka supremumu dimenzij njegovih simpleksov, $n$-dimenzionalnemu simpleksialnemu kompleksu rečemo tudi \textit{n-kompleks}. Omejili se bomo samo na končne komplekse, torej $n \in \mathbb{N}$.

Če je simpleks $\sigma$ vsebovan  v simpleksu $\tau$, mu rečemo \textit{face??} od $\tau$, rečemo mu \textit{proper face}, če $\tau\neq \sigma$. Simpleksu rečemo \textit{maksimalen simpleks}, če ni $proper face$ nobenemu drugemu simpleksu. Subkompleks $L\in K$ simplicialnega kompleksa $K$ je Simplicialni kompleks, tak da $V_L\subseteq V_K$ in $S_L\subseteq S_K$


Naj bo $\sigma = \{v_0,v_1,\ldots,v_n\}$ $n$-simpleks. Zaprt
simpleks $\bar{\sigma}$ je množica formalnih konveksnih combinacij $\mymathop{\Sigma}_{i=0}^{n}\alpha_i v_i$
pri čemer je $\alpha_i \ge 0$ za vsak $0\le i \le n$ in $\Sigma \alpha_i = 1$. Zaprt simpleks je metričen prostor z metriko

\begin{equation}
\label{eq:metrika}
d(\underset{v \in K}{\Sigma}\alpha_v v,\underset{v \in K}{\Sigma}\beta_v v) = \sqrt{\underset{v \in K}{\Sigma}(\alpha_v - \beta_v)^2}
\end{equation}

\textit{Geometrijska realizacija} $|K|$ simplicialnega kompleksa $K$ je 
množica formalnih konveksnih kombinacij $\underset{v \in K}{\Sigma}\alpha_v v$, takih da je $\{v | \alpha_v \textgreater 0\}$ simpleks v $K$.
Na $|K|$ lahko gledamo kot unijo zaprtih simpleksov $\bar{\sigma}$, za $\sigma \in K$. Množica $U\subseteq |K|$ je odprta natanko tedaj, ko je $U \cap \hat{\sigma}$ odprta, glede na metriko na $\hat{\sigma}$, za vsak $\sigma \in K$, lahko zato na $|K|$ definiramo metriko tako kot pri \ref{eq:metrika}. Če $L\subseteq K$, potem je $|L|\subseteq |K|$ zaprta podmnožica.

\textit{Polihedron/eder??} je geometrijska realizacija Simplicialnega kompleksa $|K|$, \textit{triangulacija} poliedra $X$ pa je simplicialni kompleks, katerega geometrijska realizacija je homeomorfna $X$.

Ker metrika na $|K|$ sovpada z metriko na $\bar{\sigma}$, za vsak $\sigma\in K$, sledi, da je preslikava $f$ iz $|K|$ v nek topološki prostor $X$ zvezna, natanko tedaj, ko je $f|_{\bar{\sigma}}: \bar{\sigma} \rightarrow X$ zvezna za vsak $\sigma\in K$. Tudi $H:|K|\times I \rightarrow X$ je zvezna, natanko tedaj, ko je zvezna $H|_{\si\times I}:\si\times I \rightarrow X$, za vsak $\sigma\in K$.


\textit{Simplicial preslikava} $\phi :K \rightarrow L$, med 
simplicialnima kompleksoma $K$ in $L$, je preslikava med 
ogljišči, $V_K \rightarrow V_L$, ki slika simplekse v 
simplekse. Preslikava $\phi$ inducira zvezno preslikavo med 
kompleksoma $|\phi| :|K| \rightarrow |L|$, kot $|\phi|:
\underset{v \in K}{\Sigma}\alpha_v v \mapsto
\underset{v \in K}{\Sigma}\alpha_v \phi(v)$.

    
\textit{Baricentrična subdivizija} simplicialnega kompleksa 
$K$ je simplicialni kompleks $K'$, čigar ogljišča so 
simpleksi $\sigma \in K$, simpleksi v $K'$ so pa verige 
simpleksov v $K$, urejenih z inkluzijo. Torej $\sigma' \in K'$, če $\sigma' = \{\sigma_0, \sigma_1,...,\sigma_n\}$ in $\sigma_0\subsetneq \sigma_1\subsetneq...\subsetneq\sigma_n$. \textit{Baricenter} simpleksa $\sigma \in K$ je točka $b(\sigma)=\underset{v\in \sigma}{\Sigma} \frac{v}{\#\sigma}$.

Definirajmo linearno preslikavo $S_K: |K'| \rightarrow |K|$, s predpisom $S_K(\sigma) = b(\sigma)$. Linearnost pomeni, da velja $S_K(\underset{\sigma\in \sigma'}{\Sigma} a_\sigma \sigma) =  \underset{\sigma\in \sigma'}{\Sigma} a_\sigma S_K(\sigma).$

\begin{primer}
Naj bo $K=\sigma=\{\{a\},\{b\},\{c\},\{a,b\},\{a,c\},\{b,c\},\{a,b,c\}\}$ 3-simpleks.
\\
\\
\\
\\
Slika
\\
\\
\\
\\
\\
Poglejmo si preslikavo $S_K: |K'| \rightarrow |K|$. Naj bo $x$ tako kot na sliki. %Zaradi preglednosti označimo
%$A:=\{a\}, AB:=\{a,b\},
%ABC:=\{a,b,c\}$ in analogno še za $B, C, AC, BC$.
Potem je 
$K'_x:=\textit{support(x)}=\{\{a\},\{a,b\},\{a,b,c\}\}$ in $x= \underset{\sigma\in K'_x}{\Sigma} \alpha_{i_\sigma} \sigma$. Zato

\begin{align*}
    $$
    S_K(x)&=S_K(\underset{\sigma\in K'_x}{\Sigma} \alpha_{i_{\sigma}} 
    \sigma) =  \underset{\sigma\in K'_x}{\Sigma} \alpha_{i_{\sigma}} 
    S_K(\sigma)\\
    &=\alpha_1S_K(\{a\})+\alpha_2S_K(\{a,b\})+\alpha_3S_K(\{a,b,c\}) \\ 
    &=\alpha_1a+\alpha_2\frac{a+b}{2}+\alpha_3\frac{a+b+c}{3}.
    $$
\end{align*}

Preslikava $S_K$ je očitno homeomorfizem.

\end{primer}

\nocite{*}


\bibliography{osnutek}
\bibliographystyle{plain}

\end{document}