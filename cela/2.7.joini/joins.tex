\documentclass[a4paper,12pt]{article}
\usepackage[slovene]{babel}
\usepackage[utf8]{inputenc}
\usepackage[T1]{fontenc}
\usepackage{lmodern}
\usepackage{verbatim}

\usepackage{url}
\usepackage{graphicx}
\usepackage{amsmath}
\usepackage{amsthm}
\usepackage{dsfont}
\usepackage{amssymb}
\usepackage{hyperref}



\makeatletter
\DeclareRobustCommand{\sqcdot}{\mathbin{\mathpalette\morphic@sqcdot\relax}}
\newcommand{\morphic@sqcdot}[2]{%
\sbox\z@{$\m@th#1\centerdot$}%
\ht\z@=.33333\ht\z@
\vcenter{\box\z@}%
}
\makeatother

\makeatletter
\DeclareRobustCommand{\k}{
    \mathcal{K}
}

\newcommand\mymathop[1]{\mathop{\operatorname{#1}}}

\title{Minimalni končni modeli prostorov}
\author{Filip Bezjak \\ Mentor: dr. Petar Pavešić}


\setlength\parindent{24pt}

\theoremstyle{definition}
\newtheorem{definicija}{Definicija}

\theoremstyle{plain}
\newtheorem{izrek}{Izrek}

\theoremstyle{plain}
\newtheorem{primer}{Primer}

\theoremstyle{plain}
\newtheorem{trditev}{Trditev}

\theoremstyle{plain}
\newtheorem{opomba}{Opomba}

\theoremstyle{plain}
\newtheorem{lema}{Iema}

\newenvironment{dokaz}{\begin{proof}[\bfseries\upshape\proofname]}{\end{proof}}

\begin{document}
V tem poglavju bomo definirali nekaj osnovnih konstrukcij iz algebraične topologije in jih uporabili na simplicialnih kompleksih in končnih ter splošnih topoloških prostorih

\textit{Join} Topoloških prostorov $X$ in $Y$ je topološki prostor $X\ast Y = X\times Y 
\times I /_{\sim}$, pri čemer $(x, y1, 0) \sim (x, y2, 0)$ in  $(x_1, y, 1) \sim (x2, y, 1)$. 
Torej $X\times Y\times \{0\}$ strnemo na $X$ in $X\times Y\times \{1\}$ na $Y$. Intuitivno, 
to pomeni, da vsako točko na $X$ z intervalom povežemo z vsako točko na $Y$
Posebna primera joina sta "stožec" $CX$, ki je join točke in prostora $X$, 
$$\{\bullet\}\ast X=X\times I /_{(X\times \{0\})}$$
in \textit{suspenzija} $\Sigma X$, ki je join $X$ in prostora na dveh točkah, $S^0$.

$$
\Sigma X=S^0\ast X = X\times I /_{(X\times \{0\},X\times \{1\})}
$$






\textit{Simplicialni "Join" $K\ast L$} (včasih tudi $KL$) kompleksov $K$ in $L$ z disjunktnima množicama ogljišč je kompleks

$$
K\ast L=K\cup L \cup \{\sigma \cup \tau| \sigma \in K, \tau \in L \}
$$

\begin{primer}
    simplicialni join dveh 1-simpleksov je 3 simpleks. Slika?
\end{primer}

\textit{Simplicialni stožec} $aK$ z bazo $K$ je join $K$ in ogljišča $a\notin K$
Za vsaka končna simplicialna kompleksa $K$ in $L$ velja, da je geometrijska realizacija $|K\ast L|$ homeomorfna topološkemu joinu $|K|\ast |L|$ !!!dokaz?!!.

Če je $K$ 0-kompleks z dvema ogljiščema, potem je $|K\ast L|=|K|\ast |L|=S^0\ast |L|$. $S^0\ast |L|$
imenujemo suspenzija od $L$ in oz
\end{document}