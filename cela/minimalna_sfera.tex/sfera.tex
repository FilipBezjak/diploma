\documentclass[a4paper,12pt]{article}
\usepackage[slovene]{babel}
\usepackage[utf8]{inputenc}
\usepackage[T1]{fontenc}
\usepackage{lmodern}
\usepackage{verbatim}

\usepackage{url}
\usepackage{graphicx}
\usepackage{amsmath}
\usepackage{amsthm}
\usepackage{dsfont}
\usepackage{amssymb}
\usepackage{hyperref}



\makeatletter
\DeclareRobustCommand{\sqcdot}{\mathbin{\mathpalette\morphic@sqcdot\relax}}
\newcommand{\morphic@sqcdot}[2]{%
\sbox\z@{$\m@th#1\centerdot$}%
\ht\z@=.33333\ht\z@
\vcenter{\box\z@}%
}
\makeatother

\makeatletter
\DeclareRobustCommand{\k}{
    \mathcal{K}
}

\newcommand\mymathop[1]{\mathop{\operatorname{#1}}}

\title{Minimalni končni modeli prostorov}
\author{Filip Bezjak \\ Mentor: dr. Petar Pavešić}


\setlength\parindent{24pt}

\theoremstyle{definition}
\newtheorem{definicija}{Definicija}

\theoremstyle{plain}
\newtheorem{izrek}{Izrek}

\theoremstyle{plain}
\newtheorem{primer}{Primer}

\theoremstyle{plain}
\newtheorem{posledica}{Posledica}

\theoremstyle{plain}
\newtheorem{trditev}{Trditev}

\theoremstyle{plain}
\newtheorem{opomba}{Opomba}

\theoremstyle{plain}
\newtheorem{lema}{Iema}

\newenvironment{dokaz}{\begin{proof}[\bfseries\upshape\proofname]}{\end{proof}}

\begin{document}

\maketitle

\begin{definicija}
    Končni topološki prostor je \textit{model} prostora $X$, če mu je šibko homotopsko ekvivalenten. Model je \textit{minimalen}, če ima izmed vseh modelov najmanjšo kardinalnost.
\end{definicija}

Spomnimo se, da je minimalni končni prostor, prostor brez odpravljivih točk, ker pa je vsak končen prostor homotopen svojemu jedru, sledi, da je vsak minimalni končni model prostora tudi minimalen končen prostor.

\begin{trditev}
    Končen prostor $\mathcal{S}^n(S^0)$ je končni model n-dimenzionalne sfere $S^n$ za vsak $n\geq 0$
\end{trditev}

\begin{dokaz}
    po opombi \ref{op:nekaj} velja $|\k(\mathcal{S}^n(S^0))|=|\k(S^0\circledast S^0 \circledast \cdots \circledast S^0)|=|\k(S^0)\ast
    \k(S^0) \ast \cdots \ast \k(S^0)|=|\k(S^0)|\ast
    |\k(S^0)| \ast \cdots \ast |\k(S^0)|=S^0\ast
    S^0 \ast \cdots \ast S^0$
\end{dokaz}

Zdaj bomo še dokazali, da je $\mathcal{S}^n(S^0)$ minimalni končni model za $S^n$. Še več, pokazali bomo, da ima vsak prostor, šibko homotopsko ekvivalenten $S^n$ vsaj $2n+2$ točk, če ima pa natanko $2n+2$ točk pa je homeomorfen $\k(\mathcal{S}^n(S^0))$.

\begin{definicija}
    \textit{Višina} $ht(X)$ končne delno urejene množice je ena manj kot dožina najdaljše verige v $X$. Z $#X$ pa označimo število elementov v $X$.
\end{definicija}
Dimenzija prirejenega kompleksa $\k(X)$ sovpada z $ht(X)$.


\begin{izrek}
    Naj bo $X\neq\ast$ minimalen prostor, potem ima vsaj $2ht(X)+2$ točk. Če ima natanko $2ht(X)+2$ točk, potem je homeomorfen $\mathds{S}^{ht(X)}(S^0)$
\end{izrek}    

\begin{dokaz}
    Naj bo $x_0\le x_1 \le \cdots \le x_h$ veriga dolžine $h=ht(X)$. Ker je $X$
     minimalen, $x_i$ ni odpravljiva točka za noben $0\leq i \le h$. Potem za 
     vsak $0\leq i \le h$ obstaja $y_{i+1}$, tak da $y_{i+1}\ge x_i$ in $y_{i+1}
     \ngeq x_{i+1}$. Trdimo, da so vse točke $y_i$ med seboj različne, za vsak
      $0\le i \leq h$ in da nobena ni enaka $x_j$ za noben $0\leq i \leq h$.

      Ker $y_{i+1}\geq x_i$, sledi, da $y_{i+1}\neq x_j$ za noben $j\leq i$, 
      ker velja tudi $y_{i+1}\ngeq x_{i+1}$ pa sledi, da $y_{i+1}\neq x_j$ za 
      noben $j> i$

      Če je $y_{i+1}= y_{j+1}$ za nek $i\le j$, potem je $y_{i+1}= y_{j+1}\geq
       x_j \geq x_{i+1}$, kar je pa v protislovju z predpodstavko $y_{i+1} 
       \ngeq x_{i+1}$.

       Ker je vsak končen prostor z minimumom ali z maksimumom kontraktibilen 
       in je $X\neq \ast$, minimalen prostor, sledi da $X$ nima minimuma, 
       torej mora obstajati točka $y_0\in X$, za katero velja $y_0 \ngeq x_0$.
        Zato je $y_0$ različna od drugih $2h+1$ točk in zato $#X\geq 2h+2$.

        Predpodstavimo zdaj, da ima $X$ natanko $2h+2$ točk, torej 
        $$
        X=\{x_0,x_1,\cdots x_h,y_0,y_1,\cdots y_h,\}
        $$
        Če bi bil $x_i\ge y_i$, za $0<i\leq h$ bi bilo to v nasprotju z 
        maksimalnostjo verige $x_0 <\cdots <x_h$, saj bi potem veljalo $x_{i-1} 
        < y_i < x_i$. Tudi $y_i \ngeq x_i$ za $0\leq i \leq h$, zata sta $x_i $ in $y_i$ neprimerljiva za $0\leq i \leq h$.

        Z indukcijo na $j$ pokažimo, da $y_i < x_j$ za vse $i<j$. Za $j=0$ to 
        očitno drži. Naj bo $0\leq k <h$ in recimo, da trditev drži za $j=k$, 
        dokažimo, da drži tudi za $j=k+1$. Ker $x_{k+1}$ ni navzdol odpravljiva, 
        obstaja $z$, da $z< x_{k+1}$ in $z\nleq x_k$, ker sta $x_{k+1}$ in 
        $y_{k+1}$ neprimerljiva, velja tudi $z\neq y_{k+1}$. Iz indukcijske 
        predpodstavke sledi, da je vsaka točka z izjemo $y_k$ in $y_{k+1}$ večja 
        od $x_{k+1}$ ali manjša od $x_k$. Ker $y_{k+1} \nleq x_{k+1}$, je potem
        $z=y_k$ in zato $y_k<x_{k+1}$. Dokažimo še, da $x_k\leq y_{k+1}$. Ker 
        $y_{k+1}$ ni navzdol odpravljiva, obstaja $w\in X$, da je $w<y_{k+1}$ in 
        $w\nleq x_k$. Iz indukcije in dejstva, da $y_{k+1}\ngeq x_{k+1}$, sklepamo,
        da $w=y_k$ in zato $y_k<y_{k+1}$. Za $i<k$ pa velja $y_i<x_k<x_{k+1}$ in
        $y_i<x_k<y_{k+1}$.

        Dokazali smo, da za vsak $i<j$,  velja $y_i < x_j,\ y_i < y_j,\ x_i < x_j$ in
        $x_i < y_j$ in, da sta $x_i$ in $y_i$ neprimerljiva za vsak $0\leq j \leq h$.
        To je pa urejenost v $\mathds{S}^h(S^0)$ in zato je $X$ homeomorfen 
        $\mathds{S}^h(S^0)$.

\end{dokaz}


%\begin{izrek}
 %   Vsak prostor, ki ima iste homotopske grupe kot $S^n$ (to pomeni, da mu je šhe?) ima vsaj $2n+2$ točk. Edini prostor z $2n+2$ točkami s to lastnostjo je $\mathds{S}^n(S^0)$.
%\end{izrek}

Ker je $\mathds{S}^n(S^0)$ model za $S^n$, čigar višina je enaka $n$ in ima 
$2n+2$ točk, je $\mathds{S}^n(S^0)$ minimalni model za $S^n$. Model je 
enoličen, saj je vsak model za $S^n$ na $2n+2$ točkah homeomorfen 
$\mathds{S}^n(S^0)$.

\begin{\section{Minimalni model končnega grafa}
    
\end{document}