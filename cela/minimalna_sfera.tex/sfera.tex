\documentclass[a4paper,12pt]{article}
\usepackage[slovene]{babel}
\usepackage[utf8]{inputenc}
\usepackage[T1]{fontenc}
\usepackage{lmodern}
\usepackage{verbatim}

\usepackage{url}
\usepackage{graphicx}
\usepackage{amsmath}
\usepackage{amsthm}
\usepackage{dsfont}
\usepackage{amssymb}
\usepackage{hyperref}



\makeatletter
\DeclareRobustCommand{\sqcdot}{\mathbin{\mathpalette\morphic@sqcdot\relax}}
\newcommand{\morphic@sqcdot}[2]{%
\sbox\z@{$\m@th#1\centerdot$}%
\ht\z@=.33333\ht\z@
\vcenter{\box\z@}%
}
\makeatother

\makeatletter
\DeclareRobustCommand{\k}{
    \mathcal{K}
}

\newcommand\mymathop[1]{\mathop{\operatorname{#1}}}

\title{Minimalni končni modeli prostorov}
\author{Filip Bezjak \\ Mentor: dr. Petar Pavešić}


\setlength\parindent{24pt}

\theoremstyle{definition}
\newtheorem{definicija}{Definicija}

\theoremstyle{plain}
\newtheorem{izrek}{Izrek}

\theoremstyle{plain}
\newtheorem{primer}{Primer}

\theoremstyle{plain}
\newtheorem{trditev}{Trditev}

\theoremstyle{plain}
\newtheorem{opomba}{Opomba}

\theoremstyle{plain}
\newtheorem{lema}{Iema}

\newenvironment{dokaz}{\begin{proof}[\bfseries\upshape\proofname]}{\end{proof}}

\begin{document}

\maketitle

\begin{definicija}
    Končni topološki prostor je \textit{model} prostora $X$, če mu je šibko homotopsko ekvivalenten. Model je \textit{minimalen}, če ima izmed vseh modelov najmanjšo kardinalnost.
\end{definicija}

Spomnimo se, da je minimalni končni prostor, prostor brez odpravljivih točk, ker pa je vsak končen prostor homotopen svojemu jedru, sledi, da je vsak minimalni končni model prostora tudi minimalen končen prostor.

\begin{trditev}
    Končen prostor $\mathcal{S}^n(S^0)$ je končni model n-dimenzionalne sfere $S^n$ za vsak $n\geq 0$
\end{trditev}

\begin{dokaz}
    po opombi \ref{op:nekaj} velja $|\k(\mathcal{S}^n(S^0))|=|\k(\S^0\circledast S^0 \circledast \cdots \circledast S^0)|=
\end{dokaz}
\end{document}