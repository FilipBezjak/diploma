\documentclass[a4paper,12pt]{article}
\usepackage[slovene]{babel}
\usepackage[utf8]{inputenc}
\usepackage[T1]{fontenc}
\usepackage{lmodern}
\usepackage{verbatim}

\usepackage{url}
\usepackage{graphicx}
\usepackage{amsmath}
\usepackage{amsthm}
\usepackage{dsfont}
\usepackage{amssymb}
\usepackage{hyperref}



\makeatletter
\DeclareRobustCommand{\sqcdot}{\mathbin{\mathpalette\morphic@sqcdot\relax}}
\newcommand{\morphic@sqcdot}[2]{%
\sbox\z@{$\m@th#1\centerdot$}%
\ht\z@=.33333\ht\z@
\vcenter{\box\z@}%
}
\makeatother

\makeatletter
\DeclareRobustCommand{\k}{
    \mathcal{K}
}

\newcommand\mymathop[1]{\mathop{\operatorname{#1}}}

\title{Minimalni končni modeli prostorov}
\author{Filip Bezjak \\ Mentor: dr. Petar Pavešić}


\setlength\parindent{24pt}

\theoremstyle{definition}
\newtheorem{definicija}{Definicija}

\theoremstyle{plain}
\newtheorem{izrek}{Izrek}

\theoremstyle{plain}
\newtheorem{primer}{Primer}

\theoremstyle{plain}
\newtheorem{trditev}{Trditev}

\theoremstyle{plain}
\newtheorem{opomba}{Opomba}

\theoremstyle{plain}
\newtheorem{lema}{Iema}

\newenvironment{dokaz}{\begin{proof}[\bfseries\upshape\proofname]}{\end{proof}}

\begin{document}

\maketitle


\begin{izrek}{McCordov}\label{iz:mccord}
    Naj bosta $X$ in $Y$ topološka prostora in naj bo $f:X\rightarrow Y$ zvezna. Če je zožitev
    $$
    f|_{f^{-1}}:f^{-1}(U)\rightarrow U
    $$
    Šibka homotopska ekvivalenca za vsako bazno množico $U$, potem je $f:X\rightarrow Y$  šibka homotopska ekvivalenca.
\end{izrek}

\begin{opomba}
    Izrek ne velja le za zožitev na bazne množice, ampak tudi na vsako \textit{basis like open cover}, torej za vsako pokritje, ki je baza za kako drugo topologijo.
\end{opomba}

\begin{definicija}
    Naj bo $X$ končen $T_0$ prostor. \textit{Simplicialni kompleks} $\mathcal{K}(X)$ \textit{prirejen X}, je simplicialni kompleks, čigar simpleksi so neprazne verige v $X$. Če je $f: X\rightarrow Y$ zvezna preslikava med dvema $T_0$ prostoroma. \textit{prirejena simplicialna preslikava} $\mathcal{K}(f):\mathcal{K}(X) \rightarrow \mathcal{K}(Y)$ definiramo kot $\mathcal{K}(f)(x) = f(x)$.
\end{definicija}

Vidimo, če je $f: X\rightarrow Y$ zvezna , je $\mathcal{K}(f):\mathcal{K}(X) \rightarrow \mathcal{K}(Y)$ simplicialna, saj ohranja ureditev in slika verige v verige.

\begin{lema}\label{lem:sibka}
    Naj bo $x\in X$ in naj bo $L=X\ \backslash \
    U_x\subseteq \mathcal{K}(X)$. Potem se vsak $\alpha \in \k(X)\ \backslash \ |L|$ da napisati, kot $\alpha = t\beta + (1-t)\gamma$, za $\beta \in \k(U_x), \ \gamma \in |L|$ in $0<t\leq 1$, pri čemer je $\alpha$ zvezno odvisna od $\beta, \gamma$ in $t$. $\beta, \gamma$ in $t$ so enolični.
\end{lema}
\begin{dokaz}
    $L$ je subkompleks, ki ga napenjajo ogljišča, ki niso v $U_x$. Za vsak $\alpha \in |\k(X)|\ \backslash \ |L|$, 
    $$\alpha = \sum_{i=1}^{n} \alpha_i v_i 
    = \sum_{i=1}^{r} \alpha_i u_i + \sum_{i=r+1}^{n}\alpha_i v_i,\ \text{pri čemer}\ \sum_{i=1}^{n} \alpha_i=1
    $$
    za $u_i \in U_x$ in $v_i \in X \ \backslash \ U_x$ in $\alpha_i \in \mathbb{R}$, za $r\in \{1,2, \cdots, n-1\}$ in $\forall i \in \{1,2, \cdots, n\}$. S t označimo $\sum_{i=1}^{r} \alpha_i$, torej je $1-t=\sum_{i=r+1}^{n} \alpha_i$ in $0<t\leq 1$. Potem $\beta =\sum_{i=1}^{r} \alpha_i u_i/t \in \k(U_x)$, saj je $\sum_{i=1}^{r} \alpha_i/t=1$ in podobno $\gamma=\sum_{i=r+1}^{n} 
    \alpha_i v_i/(1-t) \in \k(X \ \backslash \ U_x)$. Zveznost in enoličnost sledi iz konstrukcije.

\end{dokaz}

\begin{izrek}
    $\mathcal{K}$-\textit{McCordova} preslikava je šibka homotopska 
    ekvivalenca za vsak končen $T_0$-prostor.
\end{izrek}

\begin{dokaz}
    Definirajmo retrakcijo $r:U_x\rightarrow \{x\}$ kot 
    $r(y)=x$, za vsak $y\in X$. Ker je $x$ maksimum v 
    $U_x$, je $r\geq 1_X$, zato je po izreku 
    \ref{iz:ograje} $r\simeq 1_X$, zato je $U_x$ 
    kontraktibilna množica. Dokazali bomo, da je za vsak 
    $x\in X$, $\mu_X^{-1}(U_x)$ odprta in kontraktibilna. S 
    tem bomo pokazali, da je $\mu_X$ zvezna in da so 
    zožitve $\mu_X|_{\mu_X^{-1}(U_x)}:\mu_X^{-1}(U_x)\rightarrow 
    U_x$ šibke homotopske ekvivalence.

    Naj bo $x\in X$ in naj bo $L=X\ \backslash \
    U_x\subseteq \mathcal{K}(X)$. $L$ je torej 
    subkompleks, ki ga napenjajo ogljišča, ki niso v $U_x$. 
    Trdimo, da 
    $$
    \mu_X^{-1}(U_x)=|\mathcal{K}(X)|\ \backslash \ |L|.
    $$
    Pokažimo najprej, da $\mu_X^{-1}(U_x)\subseteq 
    |\mathcal{K}(X)|\ \backslash \ |L|$. Naj bo $\alpha \in 
    \mu_X^{-1}(U_x)$, torej je min$(\textit{support}
    (\alpha))\in U_x$, zato \textit{support}($\alpha$) vsebuje 
    ogljišče iz $U_x$, zato $\alpha \notin |L|$, torej $\alpha 
    \in |\mathcal{K}(X)|\ \backslash \ |L|$.

    Pokažimo še, da $|\mathcal{K}(X)|\ \backslash \
    |L|\subseteq \mu_X^{-1}(U_x)$. Naj $\alpha \in |\mathcal{K}(X)|\ \backslash \ |L|$
    Če  $\alpha \notin |L|$, potem obstaja $y\in 
    \textit{support}(X)$, tak, da $y \in U_x$, zato je 
    min$(\textit{support}(X))\leq y \leq x$, zato je 
    $\mu_X(\alpha) \in U_x$, zato $\alpha \in \mu_X^{-1}
    (U_x)$.
    Ker je $L$ zaprta podmnožica $\mathcal{K}(X)$, je 
    $\mu_X^{-1}(U_x)$ odprta.

    Pokažimo, da je $\mu_X^{-1}(U_x)$ kontrabilna. Prvo pokažimo, da je 
    $|\mathcal{K}(U_x)|$ krepak deformacijski retrakt 
    od $|\mathcal{K}(X)|$. Naj bo $i:|\k(U_x)|\hookrightarrow |\mathcal{K}
    (X)|\ \backslash \ |L|$ inkluzija. Če je $\alpha \in |\mathcal{K}(X)|\ 
    \backslash \ |L|$, potem je po lemi \ref{lem:sibka}  $\alpha = t\beta + 
    (1-t)\gamma$, za $\beta \in \k(U_x), \ \gamma \in |L|$ in $0<t\leq 1$. 
    Definirajmo $r:|\mathcal{K}(X)|\ \backslash \ |L|\rightarrow \k(U_x)$ 
    kot $r(\alpha)=\beta$. Ker je $\alpha$ zvezna in je zožitev $r|_{(|\mathcal{K}(X)|\ \backslash \ |L|)\cap 
    \overline{\sigma}}:(|\mathcal{K}(X)|\ \backslash \ |L|)\cap 
    \overline{\sigma} \rightarrow \overline{\sigma}$ zvezna, za vsak 
    $\sigma \in \k(X)$ , sledi in je da je $r$ zvezna. Definirajmo zdaj linearno homotopijo $H:(|\mathcal{K}(X)|\ \backslash \ |L|) \times I \rightarrow (|\mathcal{K}(X)|\ \backslash \ |L|)$ med $1_{(|\mathcal{K}(X)|\ \backslash \ |L|)}$ in $ir$ kot 
    $$
    H(\alpha,s)=(1-s)\alpha + s\beta.
    $$
    H je dobro definirana, in zvezna, saj je vsaka zožitev 
    $$
    H|_{((|\mathcal{K}(X)|\ \backslash \ |L|)\cap 
    \overline{\sigma})\times I}:((|\mathcal{K}(X)|\ \backslash \ |L|)\cap 
    \overline{\sigma})\times I \rightarrow \overline{\sigma}
    $$
    dobro definirana in zvezna, $\sigma \in \k(X)$.

    Ker je vsak element iz $U_x$ primerljiv z $x$, je $\k(U_x)$ 
    simplicialni stožec, zato je po trditvi \ref{kr neki} $|\k(U_x)|$ 
    kontraktibilen in zato je kontraktibilen tudi $\mu_X^{-1}
    (U_x)=|\mathcal{K}(X)|\ \backslash \ |L|$.
\end{dokaz}

\begin{definicija}
    Naj bo $K$ končen simplicialni kompleks. Končen $T_0$-
    prostor $\chi(K)$ prirejen k $K$ je delno urejena množica simpleksov v $K$, urejena glede na inkluzijo.
    Naj bo $phi:K\rightarrow L$ preslikava med simplicialnima kompleksoma, potem preslikavo  $\chi(phi):\chi(K)\rightarrow \chi(L)$ definiramo kot 
    $\chi(phi)(\sigma)=\varphi(\sigma)$ za vsak simpleks $\sigma \in K$
\end{definicija}

\begin{primer}
    primer iz knjige.
\end{primer}
\begin{lema}
    \label{lem:komutira}
    Naj bo $f :X\rightarrow Y$ zvezna preslikava med dvema $T_0$ prostoroma, potem naslednji diagram komutira
\end{lema}

\begin{dokaz}
    \begin{align*}
        $$
        f\mu_X(\alpha)&=f(min(\textit{support}(\alpha)))\overset{*}{=}min(f(\textit{support}(\alpha))) \\
        &=min(\textit{support}(|\k(f)(\alpha)))=\mu_Y|\k(f)|(\alpha)
        $$
    \end{align*}

    Pri čemer $*$ velja zaradi zveznosti $f$, druge enakosti pa veljajo kar po definiciji.
\end{dokaz}

Če je $K$ končen kompleks, potem je $\k(\chi(K))$ prva baricentrična subdivizija.
definirajmo $\chi$-\textit{McCordovo preslikavo} $\mu_K=\mu_{\chi(K)}S_K^{-1}: |K|\rightarrow \chi(K)$. 
Ker je kompozitum dveh šibkih homotopskih ekvivalenc tudi šibka homotopska ekvivalenca, takoj sledi naslednji izrek.

\begin{izrek}
    $\chi$-McCordova preslkava $\mu_K$ je šibka homotopska
     ekvivalenca za vsak končen simplicialni kompleks $K$.
\end{izrek}

\begin{trditev}
    Naj bo $\varphi: K\rightarrow L$ simplicialna preslikava med končnima kompleksoma. Potem naslednji diagram komutira do homotopije natančno
\end{trditev}

\begin{proof}
    Najprej poiščimo homotopijo med $|\varphi|s_K$ in $s_L|\varphi'|$
    Naj bo $S=\{\sigma_1,\sigma_2,\cdots,\sigma_r\}$ simpleks 
    v $K'$ in naj bo $\sigma_1 \subsetneq \sigma_2 \subsetneq 
    \cdots \subsetneq \sigma_r$ veriga simpleksov $K$. Naj bo $\alpha$
    točka v zaprtem simpleksu $\overline{S}$. Potem je $S_K(\alpha)
    \in \overline{\sigma_r}\subseteq |K|$ in  $|\varphi|S_K(\alpha) \in 
    \overline{\varphi_r}\subseteq |L|.$ \
    Velja pa tudi $|\varphi'|(\alpha)\in\
    {\varphi(\sigma_1),\varphi(\sigma_2),\cdots,\varphi(\sigma_r)\}$
    in potem $S_L|\varphi'|(\alpha) \in \overline{\varphi(\sigma_r)}.$ Zato 
    je linearna homotopija



    \begin{centering}
        $$
        H:|K'|\times I \rightarrow |L| \\
        H: (\alpha,t) \mapsto (1-t)|\varphi|S_K(\alpha) + tS_L|\varphi'|(\alpha)\\
        $$
    \end{centering}

Zvezna in dobro definirana in zato $|\varphi|S_K(\alpha) \simeq 
S_L|\varphi'|$. Iz leme \ref{lem:komutira} potem sledi
\begin{align*}
$$
\mu_L|\varphi|=\mu_{\chi(L)}S_L^{-1}|\varphi|&\simeq mu_{\chi(L)}|\varphi'|S_K^{-1} \\
\chi(\varphi)\mu_{\chi(K)}S_K^{-1}&=\chi(\varphi)\mu_K

$$


\end{align*}
\end{proof}

Iz lastnosti 2 od 3 in dejstva, da je preslikava, ki je homotopna šibki homotopski ekvivalenci isto šibka homotopska ekvivalenca, takoj sledi naslednja trditev.

\begin{trditev}
    Naj bo $\varphi : K\rightarrow L$ simplicialna preslikava med končnima kompleksoma, potem je $|\varphi|$ šibka homotopska ekvivalenca, natanko tedaj, ko je $\chi(\varphi)$ šibka homotopska ekvivalenca.
\end{trditev}

\section{Zanke v Hassejevem diagramu}

Pokazali bomo, kako se fundamentalna grupa končnega $T_0$ prostora izraža preko prirejenega Hassejevega diagrama.
Hassejev diagram končnega $T_0$ prostora $X$ označimo z $H(X)$, z $E(H(X))$ pa označimo množico njegovih robov.

\textit{Edge path} simplicialnega kompleksa $K$ je zaporedje
$(v_0,v_1)(v_1,v_2),...,(v_{r-1},v_{r})$ urejenih parov 
ogljišč, pri čemer je $\{v_1,v_{i+1}\}$ simpleks za vsak $i$. 
Če \textit{edge path} vsebuje dva zaporedna para 
$(v_i,v_{i+1})$ in $(v_{i+1},_v_{i+2})$ in je 
$\{v_i,v_{i+1},_v_{i+2}\}$ simpleks, potem ju lahko 
zamenjamo z parom $(v_i,v_{i+1})$ in dobimo ekvivalentno a 
krajšo pot. Za poti $(v_0,v_1)(v_1,v_2),...,(v_{r-1},v_{r})$ 
in $(u_0,u_1)(u_1,u_2),...,(u_{s-1},u_{s})$ definiramo stik 
poti.....???
Omejili se bomo na zanke, torej poti, ki se začnejo in končajo z $v_0$. Z $E(K,v_0)$ označimo množico ekvivalenčnih razredov zank z začetno točko $v_0$.....


Naj bo $(X,x_0)$ končen pointed $T_0$ prostor. Urejen par 
$e=(x,y)$ imenujemo $H-rob$ od $X$, če $(x,y)\in 
E(\mathcal{H}(\mathcal{X}))$, ali $(y,x)\in 
E(H(X))$. Točki $x$ rečem \textit{začetek} $x$ in označimo 
$x=\mathfrak{o}(e)$, točki $y$ pa \textit{konec} od $e$, 
pznačimo $\mathfrak{e}(e)=y$
\nocite{*}


\bibliography{osnutek}
\bibliographystyle{plain}

\end{document}