\documentclass[a4paper,12pt]{article}
\usepackage[slovene]{babel}
\usepackage[utf8]{inputenc}
\usepackage[T1]{fontenc}
\usepackage{lmodern}
\usepackage{verbatim}

\usepackage{url}
\usepackage{graphicx}
\usepackage{amsmath}
\usepackage{amsthm}

\title{Minimalni končni modeli prostorov}
\author{Filip Bezjak \\
Fakulteta za matematiko in fiziko \\ Oddelek za matematiko}
\date{29.11.2021}

\begin{document}

\maketitle

Moja tema sodi na področje algebraične topologije na končnih prostorih. Topologije na končnih prostorih so večkrat spregledane, saj je vsaka $T_2$ topologija
na končnem prostoru diskretna. Če pa lastnosti $T_2$ ne zahtevamo, postanejo veliko bolj 
zanimive.Poglejmo si naslednji primer: imamo prostor $X = \{a,b,c,d\}$ s topologijo 
$\{0,\{a\}, \{b\}, \{a,b,c\}, \{a,b,d\},  X\}$. Ni $T_2$, saj $c$ in $d$ nimata disjunktnih okolic.
Definirajmo preslikavo $ \varphi :[0,1] -> X$ii $ \varphi ((0,1) = a, \varphi (1) = c$.
Ta preslikava je zvezna, saj je praslika vsake odprte množice odprta. Vidimo, da je
možno interval zvezno in nekonstantno preslikati v X, torej v tem prostoru obstajajo
nekonstantne poti. \\Poglejmo še preslikavo $\psi : S^1 : X$, ki $-1 $in$1$ preslika v
$c$ in $d$, odprta loka pa v $a$ in $b$. Tudi ta preslikava je zvezna. (in Homotopsko
netrivialna?).Torej obstaja tudi nekonstantna preslikava iz krožnice v X.
Poglejmo še si naslednjo topologijo na 6 točkah $\{a\},\{b\},\{c\},\{d\},\{e\},\{f\}$.
Podobno lahko v ta prostor preslikamo trikotnik, s preslikavo ---. Topologije smo do zdaj 
navajali z eksplicitnim naštevanjem odprtih množic, ampak to hitro postane precej zamudno 
in nepregledno, predvsem pri zato se temu radi izognemo.
Za vsako točko $x$ obstaja najmanjša odprta množica $U_x$, ki jo vsebuje. Oziroma 
presek vseh odprtih množic, ki vsebujejo $x$.Ta množica je odprta, saj je topologija 
zaprta za končne preseke. Točke lahko uredimo s pravilom $ x\le y \Leftrightarrow U_x 
\subseteq  U_y$. Če točke tako uredimo, dobimo šibko urejenost. Torej refleksivno in 
tranzitivno relacijo. To je očitno. Šibka urejenost postane delna, natanko takrat, ko 
je topologija $t_0$. In disktretna, ko je Topologija $T_2$. Vse zanimivo se torej dogaja
v precepu med $T_0$ in $T_2$. Če torej prejšnji množici uredimo na tak način, dobimo To
Naslenja Hassejeva diagrama:---
Homotopski razred preslikav K --> X, [K,X] je ekvivalenčni razred, kjer sta si 2 preslikavi ekvivalentni,
če sta homotopni. Za X rečemo, da je končni model geometrijskega prostora P, Če za vsak polieder
K obstaja bijekcija med [K,P] in [K,X]. Torej če obstaja preslikava $\mu$: P--> X, ki preko kompozicije
inducira preslikavo $\mu_*$: [K,P] --> [K,X]. Preslikavi $\mu$ pravimo McCordova preslikava.
Končni model je minimalni, če ima izmed vseh modelov najmanjšo kardinalnost.
končni prostor X je model prostorq (simplicialnega
kompleksa) P, če obstaja preslikava mi: P --> X, ki preko kompozicije inducira bijekcijo med homotopskimi
razredi preslikav mi_*: [K,P]--> [K,X] za poljuben simplicialni kompleks K

$Y = \{a,b,c,d, e, f\}$\\
$\tau = \{\emptyset, X, \{a\}, \{b\}, \{a,b,c\}, \{a,b,d\}\}$


\begin{frame}
    $$X = \{a,b,c,d\}$$\\
    $$\tau = \{\emptyset, X, \{a\}, \{b\}, \{a,b,c\}, \{a,b,d\}\}$$ \\
    $$\gamma : I \rightarrow X$$
    $$  \gamma  (t)  = a, \text{ za }t \in [0,1)\text{, }  \gamma  (1) = c$$\\
    \pause

    $$\psi: S^1 \rightarrow X$$
    $$\psi(-1) = c, \psi(1) = d$$
    $$ \psi(e^{it}) = a\text{ , za } t\in (0,\pi), \psi(e^{it}) = b\text{ , za } t\in (\pi,2\pi)$$


\end{frame}
\begin{frame}
    $$Y = \{a,b,c,d, e, f\}$$
    $$ \tau = \{ \emptyset, Y , \{e\}, \{f\} , \{c,e,f\},\{d,e,f\} ,\{a, c, d, e, f \}, \{b, c, d, e, f\}\}$$
    $$\varphi: \blacktriangle \rightarrow Y$$\\
    $\varphi ,\gamma$ in $ \psi$ so zvezne. \\Poliedri se dajo nekonstantno preslikati v končne prostore.
\end{frame}

\begin{frame}
    $$\text{Naj bo } U_x \text{ najmanjša odprta množica, ki vsebuje x}$$
    $$ \forall x \in X \text{ obstaja } U_x. $$
    $$x \le y \Leftrightarrow U_x \subseteq U_y $$
    $$\text{Za } T_0 \text{ topologije je to delna urejenost, za $T_2$ pa diskretna}$$
    $$ T_0 \text{ topologije na končnih prostorih predstavimo s Hassejevimi diagrami}$$
\end{frame}

\begin{frame}
    $$X = \{a,b,c,d\}$$
    $$\tau = \{\emptyset, X, \{a\}, \{b\}, \{a,b,c\}, \{a,b,d\}\}$$
    $$U_a = \{a\}, U_b=\{b\},$$
    $$ U_c = \{a, b, c\}, U_d = \{a, b, d\}$$
    \pause
    $$Y = \{a,b,c,d, e, f\}$$
    $$ \tau = \{ \emptyset, Y , \{e\}, \{f\} , \{c,e,f\},\{d,e,f\} ,\{a, c, d, e, f \}, \{b, c, d, e, f\}\}$$
    $$U_e = \{e\}, U_f=\{f\},$$
    $$U_c = \{c, e, f\}, U_d = \{d, e, f\}$$
    $$U_a = \{a,c,f,e,d\}, U_b=\{b,c,f,e,d\}$$
\end{frame}

\end{document}