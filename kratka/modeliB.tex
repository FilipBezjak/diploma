\documentclass[70]{beamer}

\usepackage[utf8]{inputenc}
\usepackage[T1]{fontenc}
\usepackage[slovene]{babel}
\usepackage{lmodern}
\usepackage{array}     % določili <{...} in >{...} pri tabelah
\usepackage{tikz}

\usetheme{Berlin}
\usecolortheme{default}
\useinnertheme[shadows]{rounded}
\useoutertheme{infolines}
\beamertemplatenavigationsymbolsempty

% pisava
\usepackage{palatino}
\usefonttheme{serif}

\begin{document}


\title{Minimalni končni modeli prostorov}
\institute[FMF]{Fakulteta za matematiko in fiziko}
\date{29.11.2021}

\begin{frame}
   \titlepage
\end{frame}

\begin{frame}
    $$X = \{a,b,c,d\}$$\\
    $$\tau = \{\emptyset, X, \{a\}, \{b\}, \{a,b,c\}, \{a,b,d\}\}$$ \\ \pause
    $$\gamma : I \rightarrow X$$
    $$  \gamma  (t)  = a, \text{ za }t \in [0,1)\text{, }  \gamma  (1) = c$$\\
    \pause

    $$\psi: S^1 \rightarrow X$$
    $$\psi(-1) = c, \psi(1) = d$$
    $$ \psi(e^{it}) = a\text{ , za } t\in (0,\pi), \psi(e^{it}) = b\text{ , za } t\in (\pi,2\pi)$$


\end{frame}
\begin{frame}
    $$Y = \{a,b,c,d, e, f\}$$
    $$ \tau = \{ \emptyset, Y , \{e\}, \{f\} , \{c,e,f\},\{d,e,f\} ,\{a, c, d, e, f \}, \{b, c, d, e, f\}\}$$
    $$\varphi: \blacktriangle \rightarrow Y$$\\ \pause
    $\varphi ,\gamma$ in $ \psi$ so zvezne. \\ \pause
    Poliedri se dajo nekonstantno preslikati v končne prostore.
\end{frame}

\begin{frame}
    $$\text{Naj bo } U_x \text{ najmanjša odprta množica, ki vsebuje x}$$ \pause
    $$ \forall x \in X \text{ obstaja } U_x. $$ \pause
    $$x \le y \Leftrightarrow U_x \subseteq U_y $$ \pause
    $$\text{Za } T_0 \text{ topologije je to delna urejenost, za $T_2$ pa diskretna}$$ \pause
    $$ T_0 \text{ topologije na končnih prostorih predstavimo s Hassejevimi diagrami}$$
\end{frame}

\begin{frame}
    $$X = \{a,b,c,d\}$$
    $$\tau = \{\emptyset, X, \{a\}, \{b\}, \{a,b,c\}, \{a,b,d\}\}$$
    $$U_a = \{a\}, U_b=\{b\},$$
    $$ U_c = \{a, b, c\}, U_d = \{a, b, d\}$$
    \pause
    $$Y = \{a,b,c,d, e, f\}$$
    $$ \tau = \{ \emptyset, Y , \{e\}, \{f\} , \{c,e,f\},\{d,e,f\} ,\{a, c, d, e, f \}, \{b, c, d, e, f\}\}$$
    $$U_e = \{e\}, U_f=\{f\},$$
    $$U_c = \{c, e, f\}, U_d = \{d, e, f\}$$
    $$U_a = \{a,c,f,e,d\}, U_b=\{b,c,f,e,d\}$$
\end{frame}


\begin{frame}
    Homotopski razred preslikav $K \rightarrow X$ je ekvivalenčni razred \\
    $ [f] = [g] \Leftrightarrow f \cong  g$
    \pause \\
    $[K,X]$ je množica homotopskih razredov    \\ \pause
    $X$ je končni model geometrijskega prostora $P$, če za poljuben polieder K
    obstaja bijekcija $\mu_*:[K,P] \rightarrow [K,X]$\pause \\
    Model je minimalen, če ima med vsemi modeli najmanjše število točk.\\ \pause
    Minimalni modeli sfer in končnih grafov.
\end{frame}
\end{document}