\begin{definicija}
    Naj bo $K$ končen simplicialni kompleks. Končen $T_0$-
    prostor $\chi(K)$ prirejen k $K$ je delno urejena množica simpleksov v $K$, urejena glede na inkluzijo.
    Naj bo $\varphi:K\rightarrow L$ simplicialna preslikava, potem preslikavo  $\chi(\varphi):\chi(K)\rightarrow \chi(L)$ definiramo kot 
    $\chi(\varphi)(\sigma)=\varphi(\sigma)$ za vsak simpleks $\sigma \in K$
\end{definicija}


\begin{primer}
    primer preslikave
\end{primer}

\begin{lema}
    \label{lem:komutira}
    Naj bo $f :X\rightarrow Y$ zvezna preslikava med dvema $T_0$ prostoroma, potem naslednji diagram komutira

    \[\begin{tikzcd}
        {|\k(X)|}\arrow{r}{|\k(f)|} \arrow[swap]{d}{\mu_X} & {|\k(Y)|} \arrow{d}{\mu_Y} \\
       X \arrow{r}{f} & Y
       \end{tikzcd}
       \]
    
\end{lema}



\begin{dokaz}
    \begin{align*}
        $$
        f\mu_X(\alpha)&=f(\text{min}(\text{supp}(\alpha)))\overset{*}{=}\text{min}(f(\text{supp}(\alpha))) \\
        &=\text{min}(\text{supp}(|\k(f)(\alpha)))=\mu_Y|\k(f)|(\alpha)
        $$
    \end{align*}

    Pri čemer $*$ velja zaradi zveznosti $f$, druge enakosti pa veljajo kar po definiciji.
\end{dokaz}

Če je $K$ končen kompleks, potem je $\k(\chi(K))$ prva baricentrična subdivizija.
definirajmo $\chi$-\textit{McCordovo preslikavo} $\mu_K=\mu_{\chi(K)}S_K^{-1}: |K|\rightarrow \chi(K)$. 
Ker je kompozitum dveh šibkih homotopskih ekvivalenc tudi šibka homotopska ekvivalenca, takoj sledi naslednji izrek.

\begin{izrek}
    $\chi$-McCordova preslkava $\mu_K$ je šibka homotopska
     ekvivalenca za vsak končen simplicialni kompleks $K$.
\end{izrek}


\begin{trditev}
    Naj bo $\varphi: K\rightarrow L$ simplicialna preslikava med končnima kompleksoma. Potem naslednji diagram komutira do homotopije natančno
    \\
  
\[\begin{tikzcd}
    {|K|} \arrow{r}{|\varphi|} \arrow[swap]{d}{\mu_K} & {|L|} \arrow{d}{\mu_L} \\
    \chi(K) \arrow{r}{\chi(\varphi)} & \chi(L)
    \end{tikzcd}
    \]
\end{trditev}

\begin{proof}
    Najprej poiščimo homotopijo med $|\varphi|s_K$ in $s_L|\varphi'|$, kjer je $\varphi'=\k\chi(\varphi)$ preslikava med baricentričnima subdivizijama $K'$ in $L'$.
    Naj bo $S=\{\sigma_1,\sigma_2,\cdots,\sigma_r\}$ simpleks 
    v $K'$ in naj bo $\sigma_1 \subsetneq \sigma_2 \subsetneq 
    \cdots \subsetneq \sigma_r$ veriga simpleksov iz $K$. Naj bo $\alpha$
    točka v zaprtem simpleksu $\overline{S}$. Potem je $S_K(\alpha)
    \in \overline{\sigma_r}\subseteq |K|$ in  $|\varphi|S_K(\alpha) \in 
    \overline{\varphi_r}\subseteq |L|.$ \
    Velja pa tudi $|\varphi'|(\alpha)\in\
    {\varphi(\sigma_1),\varphi(\sigma_2),\cdots,\varphi(\sigma_r)\}$
    in potem $S_L|\varphi'|(\alpha) \in \overline{\varphi(\sigma_r)}.$ Zato 
    je linearna homotopija



    \begin{centering}
        $$
        H:|K'|\times I \rightarrow |L| \\
        H: (\alpha,t) \mapsto (1-t)|\varphi|S_K(\alpha) + tS_L|\varphi'|(\alpha)\\
        $$
    \end{centering}
zvezna in dobro definirana in zato $|\varphi|S_K(\alpha) \simeq 
S_L|\varphi'|$. Iz leme \ref{lem:komutira} sledi, da naslednji diagram komutira 
\[\begin{tikzcd}[row sep=large, column sep=large]
    {|\k(\chi(K))|}\arrow{r}
   {|\k(\chi(\varphi)|} \arrow[swap]{d}{\mu_{\chi(K)}} & {|\k(\chi(L))|} \arrow{d}{\mu_{\chi(L)}} \\
   \chi(K) \arrow{r}{\chi(\varphi)} & \chi(L)
   \end{tikzcd}
   \]
in zato

\begin{align*}
    $
    \mu_L|\varphi|=\mu_{\chi(L)}S_L^{-1}|\varphi| \simeq \mu_{\chi(L)}|\varphi'&|S_K^{-1} \\
    =\chi(\varphi)\mu_{\chi(K)}S_K^{-1} =\chi(\varphi)\mu_K&
    $
  \end{align*}
\end{proof}

Iz lastnosti 2 od 3 (to je treba še dokazati) in dejstva, da je preslikava, ki je homotopna šibki homotopski ekvivalenci isto šibka homotopska ekvivalenca, takoj sledi naslednja trditev.

\begin{trditev}
    Naj bo $\varphi : K\rightarrow L$ simplicialna preslikava med končnima kompleksoma, potem je $|\varphi|$ šibka homotopska ekvivalenca, natanko tedaj, ko je $\chi(\varphi)$ šibka homotopska ekvivalenca.
\end{trditev}