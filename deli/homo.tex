\documentclass[a4paper,12pt]{article}
\usepackage[slovene]{babel}
\usepackage[utf8]{inputenc}
\usepackage[T1]{fontenc}
\usepackage{lmodern}
\usepackage{verbatim}

\usepackage{url}
\usepackage{graphicx}
\usepackage{amsmath}
\usepackage{amsthm}
\usepackage{dsfont}
\usepackage{amssymb}
\usepackage{hyperref}
\usepackage{tikz-cd}


\makeatletter
\DeclareRobustCommand{\sqcdot}{\mathbin{\mathpalette\morphic@sqcdot\relax}}
\newcommand{\morphic@sqcdot}[2]{%
\sbox\z@{$\m@th#1\centerdot$}%
\ht\z@=.33333\ht\z@
\vcenter{\box\z@}%
}
\makeatother

\makeatletter
\DeclareRobustCommand{\k}{
    \mathcal{K}
}

\makeatletter
\DeclareRobustCommand{\h}{
    \mathcal{H}
}
\makeatletter
\DeclareRobustCommand{\si}{
    \bar{\sigma}
}


\makeatletter
\DeclareRobustCommand{\pot}{
    $\h-$pot
}



\newcommand\mymathop[1]{\mathop{\operatorname{#1}}}

\title{Minimalni končni modeli prostorov}
\author{Filip Bezjak \\ Mentor: dr. Petar Pavešić}


\setlength\parindent{24pt}

\theoremstyle{definition}
\newtheorem{definicija}{Definicija}

\theoremstyle{plain}
\newtheorem{izrek}{Izrek}

\theoremstyle{definition}
\newtheorem{primer}{Primer}

\theoremstyle{plain}
\newtheorem{trditev}{Trditev}

\theoremstyle{plain}
\newtheorem{posledica}{Posledica}

\theoremstyle{plain}
\newtheorem{opomba}{Opomba}

\theoremstyle{plain}
\newtheorem{lema}{Iema}

\newenvironment{dokaz}{\begin{proof}[\bfseries\upshape\proofname]}{\end{proof}}

\begin{document}


\section{Homotopska in šibka homotopska ekvivalenca}

Ena izmed glavnih nalog algebraične topologije je iskanje prostorov, ki so si 
na nek način podobni oziroma ekvivalentni. Prvi pojem podobnosti, ki ga spoznamo
je homeomorfizem. Dva prostora sta homeomorfna, če lahko enega zvezno spremenimo v drugega in pri tem prostora ne lepimo in ga ne trgamo oziroma če med njima obstaja homeomorfizem. Ekvivalenco med dvema 
prostoroma pa lahko definiramo na načine, ki so veliko širši kot homeomorfizem
Na primer torus $S^1\times B^2$ in sfera $S^1$ imata podobno obliko, vendar nista
 homeomorfna. Zato bomo definirali homotopsko in šibko homotopsko ekvivalenco, ki bosta povezali prostore s podobnimi oblikami.

Spomnimo se, \textit{homotopija} je taka družina preslikav $f_t(x):X \rightarrow Y$, da je prirejena preslikava $F(x,t)=f_t(x):X \times I \rightarrow Y$ zvezna. Rečemo, da sta preslikavi $f_0$ in $f_1$ homotopni in pišemo $f_0 \simeq f_1$


\begin{definicija}
    Preslikava  $f : X \rightarrow Y$ je \textit{homotopska ekvivalenca} prostorov $X$ in $Y$, če obstaja preslikava $g: Y\rightarrow X$, taka da
    je $f g \simeq \mathds{1}$ in $gf \simeq \mathds{1}$. Rečemo, da sta prostora $X$ in
    $Y$  \textit{homotopsko ekvivalentna}.
\end{definicija}

\begin{definicija}
    Naj bo $A \subseteq X$. Preslikavo $r : X \rightarrow A$ za katero 
    velja $r|A = \mathds{1}_A$ imenujemo \textit{retrakcija}, podprostor
     $A$ pa retrakt prostora $A$. Podprostor $A \subseteq X$ je 
     \textit{deformacijski retrakt}, če obstaja homotopija $H : X \times
      I \rightarrow X$ med $\mathds{1}_X$ in kakšno retrakcijo $r : X 
      \rightarrow A$. Homotopijo $H$ imenujemo \textit{deformacijska 
      retrakcija}. Če homotopija $H$ miruje na množici $A$ jo imenujemo 
      \textit{krepka deformacijska retrakcija} prostor $A$ pa 
      \textit{krepak deformacijski} retrakt prostora X.
\end{definicija}

Če je $A\subseteq X$ in je $A$ deformacijski retrakt od $X$, potem sta 
$X$ in $A$ homotopsko ekvivalentna prostora. Res, če je $i:A\rightarrow
 X$ inkluzija, $r:X\rightarrow A$ retrakcija, potem je $ir=\mathds{1}_A$
  in $ri\simeq\mathds{1}_X$. Eden izmed načinov, da preverimo, ali sta
   prostora $A$ in $B$ homotopsko ekvivalentna, je, da poiščemo prostor
    $X$, ki vsebuje $A$ in $B$, kot deformacijska retrakta. 

Homotopija $f_t: X\rightarrow X$, ki nam da deformacijsko retrakcijo 
prostora $X$ na podprostor $A$, ima lastnost, da velja 
$f_t|A=\mathds{1}_A$, za vse $t$. V splošnem, homotopija 
$f_t:X\rightarrow Y$, katere zožitev na podprostor $A\subseteq X$ je 
neodvisna od $t$ imenujemo \textit{homotopija relativno A?}. Krepka 
deformacijskia retrakcija $X$ na $A$ je torej homotopija med retraktom
 $r:X\rightarrow A$ in identiteto $\mathds{1}_X$, relativno $A$.


\begin{primer}
    nek primer.
\end{primer}

Šibko homotopsko ekvivalenco pa opišemo s pomočjo homotopskih grup, ki
 jih bomo definirali v nadaljevanju. Prva homotopska grupa se imenuje
  fundamentalna grupa in je grupa ekvivalenčnih razredov zank v prostoru, 
 tj. poti z enako začetno in končno točko.

Pot v prostoru $X$ je zvezna preslikava $ f: I \rightarrow X$, pri čemer je $I$ enotski interval $[0,1]$. Poti sta si homotopni, če lahko eno zvezno deformiramo v drugo, brez da bi premaknili krajišči poti.
\begin{definicija}
    
    Homotopija poti v $X$ je družina preslikav $f_t:I \rightarrow X, 0\le t \le 1$, taka da
    \begin{itemize}
        \item 
        sta krajišči $f_t(0) = x_0$ in $f_t(1) = x_1$ neodvisni od $t$ in
%t po homotopiji, s po intervalu poti
        \item 
        je prirejena preslikava $F:I\times X \rightarrow X$ definirana s $F(s,t) = f_t(s)$ zvezna.
    \end{itemize}
    Za poti $f_1$ in $f_0$, ki sta povezani s homotopiijo $f_t$ rečemo, da sta homotopni in označimo $f_1 \simeq f_0$.
\end{definicija}



\begin{trditev}
    Relacija homotopije na poteh s fiksnima krajiščema je ekvivalenčna relacija za vsak topološki prostor.
\end{izrek}

\begin{dokaz}
    Preveriti moramo 3 lastnosti ekvivalenčnih relacij, refleksivnost, simetričnost in tranzitivnost.

    Najprej preverimo refleksivnost. Naj bo $f : I \rightarrow X$ pot v prostoru $X$. Homotopijo definiramo kot $f_t(s)=f(s)$.

    Naj velja $f \simeq g$ in naj bo $f_t(s)$ homotopija med $f$ in $g$, 
    torej $f_0=f$ in $f_1=g$. Homotopijo med $g$ in $f$ definiramo kot 
    $g_t(s)=f_{1-t}(s)$. Velja $g_0=f_1=g$ in $g_1=f_0=f$ in ker je 
    $g_t(s)$ kompozitum zveznih preslikav, je zvezna. Sledi, da je 
    relacija homotopije simetrična.

    Naj bodo $f, g \text{ in } h$ poti v $X$ in naj velja $f \simeq g$ 
    in $g \simeq h$. in naj bo $f_t(s)$ homotopija med $f$ in $g$ in 
    $g_t(s)$ homotopija med $g$ in $h$. Definirajmo 
    $$h_t(s)=\begin{cases}
        f_{2t}(s), & t \in [0,\frac{1}{2}] \\
        g_{2t-1}(s), & t \in [\frac{1}{2},1]
    \end{cases}
    $$
    Velja $h_0=f_0=f$ in $h_1=g_1=h$, ker je $h_t(s)$ sestavljena iz dveh zveznih poti, ki se ujemata na preseku, je zvezna, sledi, da je relacija tranzitivna.
\end{dokaz}

Za poljubni poti $f,g : I \rightarrow X$, za kateri velja $f(1) = g(0)$ 
lahko definiramo njun stik $f\sqcdot g$, ki preteče $f$ in $g$ z dvojno 
hitrostjo v enotskem intervalu.
$$ f\sqcdot g(s) \begin{cases}
    f(2s), &s \in [0,\frac{1}{2}] \\
    g(2s-1), & s \in [\frac{1}{2},1]
\end{cases}
$$

Če se omejimo samo na poti $f:I \rightarrow X$ z enako začetno in končno točko $f(0) = f(1) = x_0$, govorimo o zankah, za $x_0$ pa rečemo, da je bazna točka.
Množico vseh homotopskih razredov $[f]$, z bazno točko $x_0$ označimo z $\pi_1(X,x_0)$.

\begin{izrek}
    $\pi_1(X,x_0)$ opremljena s produktom $[f][g] = [f\sqcdot g]$ je grupa.
\end{izrek}

\begin{dokaz}
    Najprej preverimo dobro definiranost produkta. Naj velja $[f]=[f']$, preko homotopije $f_t$ in $[g]=[g']$ preko $g_t$. Potem sta $f\sqcdot g$ in $f'\sqcdot g'$ homotopni preko
    $h_t(s) =f_t \sqcdot g_t$. Vidimo, da $h_0=f_0 \sqcdot g_0=f \sqcdot g$ in $h_1=f_1 \sqcdot g_1=f'\sqcdot g'$. Ker je $f_t(1)=g_t(0)$ za vsak t in sta $ft(s)$ in $g_t(s)$ zvezni, sledi, da je tudi $h_t(s)$ zvezna, torej velja [$f\sqcdot g$]=[$f'\sqcdot g'$].

    Definirajmo še \textit{reparametrizacijo} poti $f$ kot kompozitum $f 
    \phi$, kjer je $\phi: I \rightarrow I$ neka zvezna preslikava, za 
    katero velja $\phi(0)= 0$ in $\phi(1)=1$. Reparametrizacija poti 
    ohranja homotopski razred, saj sta $f\phi$ in $f$ povezani preko 
    $f\phi_t$, pri čemer je $\phi_t(s)=(1-t)\phi(s)+ts$. Vidimo, da 
    $\phi_t(s)$ leži med $\phi(s)$ in $s$, torej na $I$, zato je 
    $f\phi_t(s)$ dobro definirana.

    Naj bodo $f$, $g$ in $h$ poti v $X$ in naj bo $f(1)=g(0)$ in 
    $g(1)=h(0)$, potem sta oba stika $(f\sqcdot g) \sqcdot h$ in 
    $f\sqcdot (g \sqcdot h)$ definirana, $(f\sqcdot g) \sqcdot h$ pa je
     reparametrizacija $f\sqcdot (g \sqcdot h)$ preko odsekoma linearne 
     funkcije
    $$
    \phi(s)=\begin{cases}
        \frac{s}{2}, &s \in [0,\frac{1}{2}] \\
        s, & s \in [\frac{1}{2},\frac{3}{4}]\\
        2s, & s \in [\frac{3}{4},1],
    \end{cases}
    $$ zato sta poti homotopni, torej je množenje v $\pi_1(X,x_0)$ asociativno.

    Naj bo $f$ pot v $X$ in naj bo $c$ konstantna pot definirana s $c(s)=f(1)$, $fc$ je reparametrizacija $f$ preko 
    $$\phi(s)=\begin{cases}
        2s, &s \in [0,\frac{1}{2}] \\
        1, & s \in [\frac{1}{2},1],\\
        \end{cases}
    $$, zato velja $fc\simeq f$, podobno velja tudi  $cf\simeq f$, kjer je $c$ konstantna pot $c(s)=f(0)$. Sklepamo, da je $c(s)=x_0$ dvostranska enota v grupi  $\pi_1(X,x_0)$

    Inverz poti $f$ definiramo kot $\bar{f}(s)=f(1-s)$. Definirajmo $h_t=f_t\sqcdot \bar{f_t}$, pri čemer je 
    $$
    f_t(s)=
    \begin{cases}
        f(s), &s \in [0,1-t] \\
        f(1-t), & s \in [1-t,1] \\
        \end{cases}
$$
Ker je $h_0=f\sqcdot \bar{f}$ in $h_1=f(0)=c$, sledi, da je $f\sqcdot \bar{f}$ homotopna konstantni poti v $x_0$. Če $f$ zamenjamo z $\bar{f}$, sledi, da $\bar{f}\sqcdot f\simeq c$, zato je $[\bar{f}]$ obojestranski inverz od $[f]$.
\end{dokaz}

Kot že povedano, tej grupi pravimo fundamentalna grupa prostora $X$, z bazno točko $x_0$. $\pi_1(X,x_0)$ je prva v zaporedju analogogno definiranih homotopskih grup $\pi_n(X,x_0)$, pri katerih namesto iz $I$ slikamo iz $n$-dimenzionalne kocke $I^n$.


Naj bo $I^n$ $n$-dimenzionalna kocka. Rob $\partial I^n \text{ od } I^n$ je podprostor točk pri katerih je vsaj ena koordinata enaka $1$ ali $0$. Definirajmo $\pi_n(X,x_0)$, množico homotopskih razredov preslikav $f:(I^n,\partial I^n) \rightarrow (X,x_0)$ pri čemer velja $f(\partial I^n) = x_0$.


Za $n\ge 2$ posplošimo stik definiran pri fundamentalni grupi.
$$ f\sqcdot g(s) \begin{cases}
    f(2s_1,s_2,\ldots,s_n), & s_1\in [0,\frac{1}{2}] \\
    g(2s_1-1,s_2,\ldots,s_n), & s_1 \in [\frac{1}{2},1]
\end{cases}
$$

\begin{izrek}
    $\pi_n(X,x_0)$ opremljena s produktom $[f][g] = [f\sqcdot g]$ je grupa za vsak $n \in \mathds{N}$.
\end{izrek}

Dokaz te trditve je enak dokazu za $n=1$, saj je v stik poti vpletena le prva komponenta poti.

\begin{definicija}
    Topološka prostora sta \textit{šibko homotopsko ekvivalentna}, če so njune homotopske grupe izomorfne za vsak $n \in \mathds{N}$.
\end{definicija}

Homeomorfni prostori so homotopsko ekvivalentni, homotopsko ekvivalentni prostori so si tudi šibsko homotopsko ekvivalentni. Obratno v splošnem ne velja.

\begin{definicija}
    Preslikava je \textit{šibka homotopska ekvivalenca}, če preko kompozicije inducira izomorfizem na vse homotopske grupe.
\end{definicija}
Če med prostoroma obstaja šibka homotopska ekvivalenca, potem sta prostora šibko homotopsko ekvivalentna.

\begin{trditev}
    
\end{trditev}
