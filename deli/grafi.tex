\documentclass[a4paper,12pt]{article}
\usepackage[slovene]{babel}
\usepackage[utf8]{inputenc}
\usepackage[T1]{fontenc}
\usepackage{lmodern}
\usepackage{verbatim}

\usepackage{url}
\usepackage{graphicx}
\usepackage{amsmath}
\usepackage{amsthm}
\usepackage{dsfont}
\usepackage{amssymb}
\usepackage{hyperref}
\usepackage{tikz-cd}


\makeatletter
\DeclareRobustCommand{\sqcdot}{\mathbin{\mathpalette\morphic@sqcdot\relax}}
\newcommand{\morphic@sqcdot}[2]{%
\sbox\z@{$\m@th#1\centerdot$}%
\ht\z@=.33333\ht\z@
\vcenter{\box\z@}%
}
\makeatother

\makeatletter
\DeclareRobustCommand{\k}{
    \mathcal{K}
}

\makeatletter
\DeclareRobustCommand{\h}{
    \mathcal{H}
}
\makeatletter
\DeclareRobustCommand{\si}{
    \bar{\sigma}
}


\makeatletter
\DeclareRobustCommand{\pot}{
    $\h-$pot
}



\newcommand\mymathop[1]{\mathop{\operatorname{#1}}}

\title{Minimalni končni modeli prostorov}
\author{Filip Bezjak \\ Mentor: dr. Petar Pavešić}


\setlength\parindent{24pt}

\theoremstyle{definition}
\newtheorem{definicija}{Definicija}

\theoremstyle{plain}
\newtheorem{izrek}{Izrek}

\theoremstyle{definition}
\newtheorem{primer}{Primer}

\theoremstyle{plain}
\newtheorem{trditev}{Trditev}

\theoremstyle{plain}
\newtheorem{posledica}{Posledica}

\theoremstyle{plain}
\newtheorem{opomba}{Opomba}

\theoremstyle{plain}
\newtheorem{lema}{Iema}

\newenvironment{dokaz}{\begin{proof}[\bfseries\upshape\proofname]}{\end{proof}}

\begin{document}


\section{Minimalni modeli končnih grafov}

Graf v topologiji je topološki prostor, ki ga dobimo iz običajnega grafa 
$G=(E,V)$, če oglišča iz $V$ zamenjamo s točkami in vsako povezavo $e\in 
E$, $e=xy$, za $x,y\in V$, zamenjamo z enotskim intervalom, pri čemer 
identificiramo $0$ in $x$ ter $1$ in $y$.
Graf je torej geometrijska realizacija enodimenzionalnega simplicialnega 
kompleksa. Graf je končen, če ima končno mnogo oglišč.


\begin{definicija}
    Naj bo $K$ simplicialni kompleks dimenzije $n$ in naj $s_m$ označuje število $m-$simpleksov v $K$
    \textit{Eulerjeva karakteristika} $\chi(X)$ simplicialnega kompleksa $K$, je  $\chi(X)=\sum_{i=0}^n (-1)^m s_m$
\end{definicija}

Če se omejimo na enodimenzionalne simplicialne komplekse, potem ta definicija sovpada z definicijo eulerjeve karakteristike grafa, ki je definirana kot $V-E$, kjer je $V$ število oglišč, $E$ pa število robov v grafu. Eulerjeva karakteristika dreves je enaka 1.

\begin{lema}
    Naj bo $G$ končen graf, in naj bo $e=\{u,v\}$ povezava v grafu in $u\neq v$, potem se vsaka homotopija 
\end{lema}

\begin{dokaz}
    dokazali bomo v treh delih 

    Prvo dokažimo, da je je $G\times \{0\}\cup e\times I$ deformacijski retrakt od $G\times I$. Nato bomo dokazali, da se vsa
\end{dokaz}
\begin{trditev}
    $G\times \{0\}\cup e\times I$ deformacijski retrakt od $G\times I$.
\end{trditev}


\begin{trditev}
    Naj bo $G$ končen graf, in naj bo $e=\{u,v\}$ povezava v grafu in $u\neq v$. potem je $G/_e$ deformacijski retrakt od $G$.
\end{trditev}

\begin{posledica}
    Naj bo $G$ končen graf, in naj bo $T$ vpeto drevo. Potem sta prostora $G$ in $G/_T$ homotopsko ekvivalentna.
\end{posledica}

Naj bo $T$ vpeto drevo grafa $G$, $\chi(T)=1$. $G$ dobimo iz $T$ tako, da mu dodajamo povezave 
oziroma $1-$simplekse, zato je $\chi(G)$ enako $1-n$, kjer je $n$ število povezav v $G$, ki niso v 
$T$. Prostor $G/_e$ dobimo iz $G$, tako da krajišči povezave zlepimo v eno točko, povezavo pa 
izbrišemo, torej ima nov prostor eno oglišče in eno povezavo manj, zato se eulerjeva karakteristika 
ohranja. $G/_T$ je prostor z enim ogliščem $x_0$ in $n$ povezavami, ki se začnejo in končajo v istem 
oglišču in je homeomorfen šopu $n=1-\chi(G)$ krožnic, kar označimo z $\vee_{i=1}^{n}S^1$.

Minimalni model grafa je torej enak minimalnemu modelu  $\vee_{i=1}^{n}S^1$. Poglejmo si 
fundamentalno grupo  $\pi_i(\vee_{i=1}^{n}S^1,x_0)$, kjer je $x_0$ točka, v kateri so krožnice 
staknjene. Vsak ekvivalenčni razred zank predstavimo z zaporedjem krožnic, ki jih prepotuje in s 
smerjo v kateri gre čez krožnico. Če s $s_i$ označimo $i-$to krožnico, potem lahko ekvivalenčni 
zank  predstavimo kot zaporedje $s_{i_1}^{j_1}s_{i_2}^{j_2}\cdots s_{i_m}^{j_m}$, za $m\in N$ in 
$j\in \{-1,1\}.$ Tu $j$ označuje smer po kateri se premikamo po krožnici. Stik zank 
$s_{i_1}^{j_1}\cdots s_{i_m}^{j_m}$ in $s_{k_1}^{l_1}\cdots s_{k_m'}^{l_h}$ pa je ekvivalenten 
stiku zaporedij $s_{i_1}^{j_1}\cdots s_{i_m}^{j_m}s_{k_1}^{l_1}\cdots s_{k_m'}^{l_h}$.
Seveda velja $s_i s_i^{-1}=s_i^{-1}s_i=1$, kjer $1$ predstavlja trivialno zanko. 
    Torej vidimo, da je fundamentalna grupa $\pi_i(\vee_{i=1}^{n}S^1,x_0)$ enaka prosti grupi z $n$
generatorji, ki jo označimo z $F_n$.
