\documentclass[a4paper,12pt]{article}
\usepackage[slovene]{babel}
\usepackage[utf8]{inputenc}
\usepackage[T1]{fontenc}
\usepackage{lmodern}
\usepackage{verbatim}

\usepackage{url}
\usepackage{graphicx}
\usepackage{amsmath}
\usepackage{amsthm}
\usepackage{dsfont}
\usepackage{amssymb}
\usepackage{hyperref}
\usepackage{tikz-cd}


\makeatletter
\DeclareRobustCommand{\sqcdot}{\mathbin{\mathpalette\morphic@sqcdot\relax}}
\newcommand{\morphic@sqcdot}[2]{%
\sbox\z@{$\m@th\#1\centerdot$}%
\ht\z@=.33333\ht\z@
\vcenter{\box\z@}%
}
\makeatother

\makeatletter
\DeclareRobustCommand{\k}{
    \mathcal{K}
}

\makeatletter
\DeclareRobustCommand{\h}{
    \mathcal{H}
}
\makeatletter
\DeclareRobustCommand{\si}{
    \bar{\sigma}
}


\DeclareMathOperator*{\supp}{supp}



\DeclareMathOperator*{\htt}{ht}


\makeatletter
\DeclareRobustCommand{\pot}{
    $\h-$pot
}



\newcommand\mymathop[1]{\mathop{\operatorname{\#1}}}

\title{Minimalni končni modeli prostorov}
\author{Filip Bezjak \\ Mentor: dr. Petar Pavešić}


\setlength\parindent{24pt}

\theoremstyle{definition}
\newtheorem{definicija}{Definicija}

\theoremstyle{plain}
\newtheorem{izrek}{Izrek}

\theoremstyle{definition}
\newtheorem{primer}{Primer}

\theoremstyle{plain}
\newtheorem{trditev}{Trditev}

\theoremstyle{plain}
\newtheorem{posledica}{Posledica}

\theoremstyle{plain}
\newtheorem{opomba}{Opomba}

\theoremstyle{plain}
\newtheorem{lema}{Iema}

\newenvironment{dokaz}{\begin{proof}[\bfseries\upshape\proofname]}{\end{proof}}

\begin{document}


\section{Minimalni modeli končnih grafov}

Graf v topologiji je topološki prostor, ki ga dobimo iz običajnega grafa 
$G=(E,V)$, če oglišča iz $V$ zamenjamo s točkami in vsako povezavo $e\in 
E$, $e=xy$, za $x,y\in V$, zamenjamo z enotskim intervalom, pri čemer 
identificiramo $0$ in $x$ ter $1$ in $y$.
Graf je torej geometrijska realizacija enodimenzionalnega simplicialnega 
kompleksa. Graf je končen, če ima končno mnogo oglišč. 

Brez dokaza bomo upoštevali dejstvo, da je za enodimenzionalne poliedre 
šibki homotopski tip odvisen le od fundamente grupe, kar pomeni, da sta 
sta prostora šibko homotopsko ekvivalentna, natanko tedaj, ko imata 
izomorfno fundamentalno grupo.


\begin{definicija}
    Naj bo $K$ simplicialni kompleks dimenzije $n$ in naj $s_m$ označuje število $m-$simpleksov v $K$.
    \textit{Eulerjeva karakteristika} $\chi(K)$ simplicialnega kompleksa $K$, je  $\chi(X)=\sum\limits_{i=0}^n (-1)^m s_m$
\end{definicija}

Če se omejimo na enodimenzionalne simplicialne komplekse oziroma grafe, 
potem je eulerjeva karakteristika enaka $V-E$, kjer je $V$ število 
oglišč, $E$ pa 
število robov v grafu. Eulerjeva karakteristika dreves je enaka 1.

\begin{trditev}
    Naj bo $G$ topološki graf in $e$ njegova povezava. Potem je
    $G\times \{0\}\cup e\times I$ deformacijski retrakt od $G\times I$.
\end{trditev}

\begin{dokaz}
    Obstaja retrakcija $r: I\times I \rightarrow I\times \{0\} \cup \partial I \times I$, naprimer radialna projekcija iz točke $(1/2,2)\in I\times \mathbb{R}$. Ta retrakcija nam porodi deformacijsko retrakcijo $r_t=tr+(1-t)Id$. Ta deformacijska retrakcija nam porodi deformacijsko retrakcijo $G\times I \rightarrow G$
\end{dokaz}

\begin{trditev}
    Naj bo $G$ končen topološki graf in  $e=\{u,v\}$ povezava v grafu za $u\neq v$, potem se vsak par preslikav $G\times \{0\}\rightarrow G$ in $e\times I \rightarrow G$, ki sovpada na $e\times \{0\}$, da razširiti do preslikave $G\times I \rightarrow G$
\end{trditev}

\begin{dokaz}
    Ker je $e$ zaprt v $G$, lahko preslikavi združimo v preslikavo $G\times \{0\}\cup e\times I\rightarrow G$, ki je zvezna, saj je zvezna na zaprtih podprostorih $G\times \{0\}$ in $e\times I$. Če to komponiramo z retrakcijo $G\times I \rightarrow G\times \{0\}\cup e\times I$, dobimo razširitev $G\times I \rightarrow G.$
\end{dokaz}

Recimo, da imamo preslikavo $f_0:G\rightarrow G$ in homotopijo $f_t:e\rightarrow G$, preslikave $f_0|A$, potem nam ta trditev pove, da lahko to homotopijo razširimo do homotopije $f_t:G\rightarrow G$ dane preslikave $f_0$.

\begin{trditev}
    Naj bo $G$ končen topološki graf in  $e=\{u,v\}$ povezava v grafu za $u\neq v$. Potem sta prostora $G$ in $G/_e$ homotopsko ekvivalentna.
\end{trditev}

\begin{dokaz}
    Naj bo $i:e\rightarrow G$ inkluzija. Povezava $e$ je kontraktibilna, torej obstaja homotopija $r_t:e \rightarrow e$ in $r_0=id_e$. Naj bo $f_t:G\rightarrow G$ razširitev homotopije $ir_t$ in naj velja $f_0=id_G$.

    Ker $f_t(e)\subseteq e$, kompozicija $qf_t:G\rightarrow G/e$ slika $e$ v točko, zato obstaja $\bar{f}_t:G/_e\rightarrow G/_e$, taka da velja $gf_t=\bar{f}_tq$. Pri $t=1$ je $f_1(e)$ enaka točki, v katero se $e$ kontrakira. Zato $f_1$ inducira preslikavo $g:G/_e \rightarrow G$, da velja $gq=f_1$. Ker velja $qg(\bar{x})=qgq(x)=qf_1(x)=\bar{f}_1 q(x)=\bar{f}_1(\bar{x})$, sledi, da je $qg=\bar{f}$. Zato sta $q$ in $g$ homotopska inverza, saj je $gq=f_1\simeq f_0=id_G$ in $qg=\bar{f}_1\simeq \bar{f}_0 = id_{G/e}$
\end{dokaz}

\begin{posledica}
    Naj bo $G$ končen graf, in naj bo $T$ vpeto drevo. Potem sta 
    prostora $G$ in $G/_T$ homotopsko ekvivalentna.
\end{posledica}

Naj bo $T$ vpeto drevo grafa $G$. Velja $\chi(T)=1$. $G$ dobimo iz $T$ tako, da mu dodajamo povezave 
oziroma $1-$simplekse, zato $\chi(G)=1-n$, kjer je $n$ število povezav v $G$, ki niso v 
$T$. Prostor $G/_e$ dobimo iz $G$, tako da krajišči povezave zlepimo v eno točko, povezavo pa 
izbrišemo, torej ima nov prostor eno oglišče in eno povezavo manj, zato se eulerjeva karakteristika 
ohranja. $G/_T$ je prostor z enim ogliščem $x_0$ in $n$ povezavami, ki se začnejo in končajo v istem 
oglišču, torej je homeomorfen šopu $n=1-\chi(G)$ krožnic, kar označimo z $\bigvee\limits_{i=1}^{n}S^1$.

\begin{posledica}
    \label{pos:karakteristika}
    Šibki homotopski tip grafa je odvisen le od eulerjeve karakteristike
\end{posledica}

Minimalni model grafa je torej enak minimalnemu modelu  $\bigvee\limits_{i=1}^{n}S^1$. Poglejmo si 
fundamentalno grupo  $\pi_1(\bigvee\limits_{i=1}^{n}S^1,x_0)$, kjer je $x_0$ točka, v kateri so krožnice 
staknjene. Vsak ekvivalenčni razred zank predstavimo z zaporedjem krožnic, ki jih prepotuje in s 
smerjo v kateri gre čez krožnico. Če s $s_i$ označimo $i-$to krožnico, potem lahko ekvivalenčni 
zank  predstavimo kot zaporedje $s_{i_1}^{j_1}s_{i_2}^{j_2}\cdots s_{i_m}^{j_m}$, za $m\in N$ in 
$j\in \{-1,1\}.$ Tu $j$ označuje smer po kateri se premikamo po krožnici. Stik zank 
$s_{i_1}^{j_1}\cdots s_{i_m}^{j_m}$ in $s_{k_1}^{l_1}\cdots s_{k_m'}^{l_h}$ pa je ekvivalenten 
stiku zaporedij $s_{i_1}^{j_1}\cdots s_{i_m}^{j_m}s_{k_1}^{l_1}\cdots s_{k_m'}^{l_h}$.
Seveda velja $s_i s_i^{-1}=s_i^{-1}s_i=1$, kjer $1$ predstavlja trivialno zanko. 
Torej vidimo, da je fundamentalna grupa $\pi_1(\bigvee\limits_{i=1}^{n}S^1,x_0)$ enaka prosti grupi z $n$
generatorji, ki jo označimo z $F_n$.

\begin{trditev}
    Naj bo $X$ povezan topološki prostor in naj $x,x_0\in X, x\neq x_0$, taka da $x$ ni niti maksimalen, niti minimalen. Potem inkluzija asociiranih simplicialnih kompleksov $\k(X\backslash\{x\})\subseteq \k(X)$ inducira epimorfizem 
    $$
i_\star:E(\k(X\backslash\{x\}),x_0)\rRghtarrow E(\k(X),x_0)
    $$
    med njunima $edgepath$ grupama.
\end{trditev}

\begin{dokaz}
    Pokazati moramo, da je vsaka lomljenka v \k(X) z izhodiščem $x_0$ ekvivalentna neki drugi lomljenki, ki ne gre skozi $x$.
    Recimo, da je $y\leq x$ in je (y,x)(x,z) lomljenka v $\k(X)$. Če je $x\leq z$, potem je $(y,z)(x,z)\equiv(y,z)$, saj je $\{x,y,z\}$ simpleks. Če je pa $z< X$, potem obstaja $w>x$, saj $x$ ni maksimalen. Zato je $(y,x)(x,z)\equiv(y,x)(x,w)(w,x)(x,z)\equiv (y,w)(w,z)$. Če je $y\geq x$ je dokaz analogen.
\end{dokaz}

Če povezanemu prostoru $X$ postopoma odstranjujemo točke, ki niso minimalne ali maksimalne, v vsakem koraku dobimo epimorfizem med fundamentalnima grupama, ki pa ni nujno izomorfizem, saj imamo lahko v $\k(X\backslash\{x\})$ dve neekvivalentni zanki, ki sta v $\k(X)$ ekvivalentni.



\begin{posledica}
    Naj bo $X$ povezan končen $T_0$ prostor. Potem obstaja povezan $T_0$ prostor $Y\subseteq X$, višine kvečjemu $1$ in epimorfizem iz $\pi_1(Y,x)$ v $\pi_1(X,x)$.
\end{posledica}

\begin{trditev}
    Naj bo $n\in\mathbb{N}$. Če je $X$ minimalen model za 
    $\bigvee\limits_{i=1}^{n}S^1$, potem $\htt(X)=1$.
\end{trditev}

\begin{dokaz}
    Naj bo $X$ minimalen model za $\bigvee\limits_{i=1}^{n}S^1$. Potem obstaja povezan $T_0-$podprostor $Y\subseteq X$ višine $1$ in epimorfizem iz $\pi_1(Y,x)$ v $\pi_1(X,x)=F_n$.

    Ker je $\htt(Y)=1$, je $Y$ model za grafa, torej $\pi_1(Y,x)=F_m$, za $m\geq n$.

    V $\h(Y)$ imamo $m$ robov, ki niso v vpetem drevesu prirejenega grafa $\k(Y)$, zato lahko odstranimo $m-n$ robov iz $\h(Y)$ tako, da ostane povezan in dobljen prostor $Z$ je model za $\bigvee\limits_{i=1}^{n}S^1$.

    Velja $\#Z=\#Y\leq \#X$, ampak ker je $X$ končen minimalen model, mora veljati $\#X\leq\#Z$ in zato $X=Z$, torej je višina $X$ enaka 1.
\end{dokaz}

Naj bo $X$ minimalen model za $\bigvee\limits_{i=1}^{n}S^1$. Vpeljimo naslednje oznake, $i:=\#\{y\in X| y \text{je minimalen}\}$ in $j:=\#\{y\in X| y \text{je maksimalen}\}$. Potem $\#X=i+j$ in $\#E(\h(X))\leq ij$. Ker je $\chi(X)=1-n$, velja $n\leq 1 - (i+j) + ij=(i-1)(j-1)$. Navedimo glavni izrek razdelka.

\begin{izrek}
    Naj bo $n\in\mathbb{N}$. Končen $T_0-$prostor $X$ je minimalni model za  $\bigvee\limits_{i=1}^{n}S^1$ natanko tedaj, ko je $\htt(X)=1$, $\#X=\min\{i+j|(i-j)(i-j)\geq n\}$ in $\#E(\h(X))= \#X + n -1.$
\end{izrek}

\begin{dokaz}
    Pokazali smo že, da če je $X$ minimalen model za $\bigvee\limits_{i=1}^{n}S^1$, potem $\htt(X)=1$ in $\#X=\min\{i+j|(i-j)(i-j)\geq n\}$. 
    Če sta $i$ in $j$ taka, da $(i-j)(i-j)\geq n$, potem definiramo prostor $Y=\{x_1,x_2,...x_i,y_1,y_2,...y_j\}$ z ureditvijo $y_k<x_l$ 
    za vse $k$ in $l$, ki je model za $\bigvee\limits_{i=1}^{(i-1)(j-1)}S^1$. Potem lahko odstranimo $(i-1)(j-1)-n$ robov iz $\h(X)$ in tako dobimo povezan prostor kardinalnosti $i+j$, ki je model za $\bigvee\limits_{i=1}^{n}S^1$. Torej $\#X\leq \#Y=i+j$. To drži za vsaka $i$ in $j$, za katera velja $n\leq (i-1)(j-1)$, zato $\#X=\min\{i+j|(i-j)(i-j)\geq n\}$, zaradi minimalnosti $X$. $\#E(\h(X))= \#X + n -1$ pa sledi, ker $\chi(X)= 1-n$.

    Dokažimo še v drugo smer. Naj velja  $\htt(X)=1$, $\#X=\min\{i+j|(i-j)(i-j)\geq n\}$ in $\#E(\h(X))= \#X + n -1.$ Dokazati moramo le, da je $X$ povezan, saj potem iz prve in tretje predpodstavke sledi, da je $X$ model za  $\bigvee\limits_{i=1}^{n}S^1$, iz druge pa sledi, da je model minimalen.

    Naj bodo $X_l$ komponente povezanosti od $X$, za $1\leq l \leq k$. Z $M_l$ označimo množico maksimalnih elementov v $X_l$, z $m_l$ pa $X\backslash M_l$. Naj $i=\# \sum\limits_{l=1}^k \# M_l$ in $j==\# \sum\limits_{l=1}^k \# m_l$. Ker $i+j=\# X =\min \{s+t|(s-1)(t-1)\geq n\}$, sledi, da $(i-2)(j-1)<n=\# E(\h(X))-\# X +1=E(\h(X)) -(i+j) +1$ in zato $ij - E(\h(X))<j-1$. To pomeni, da se $\k(X)$ od polnega dvodelnega grafa $(\cup m_l,\cup M_l)$
    razlikuje v manj kot $j-1$ robovih. Ker za $r\neq l$ ne obstaja povezava med ogliščem v $m_l$ in ogliščem v $M_r$, velja
    $$
        j-1>\sum\limits_{l=1}^{k}\# M(j- \# m_l)\geq \sum\limits_{l=1}^{k}(j- \# m_l)=(k-1)j.
    $$

    Zato $k=1$ in dokaz je končan.
\end{dokaz}

\end{document}