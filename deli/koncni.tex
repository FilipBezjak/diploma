
\section{Končni topološki prostori in delno urejene množice}

\textit{Končni topološki prostor} je topološki prostor s končno mnogo točkami, 
\textit{šibko urejena} množica je množica s tranzitivno in z refleksivno relacijo. Če je relacija še antisimetrična, dobimo \textit{delno} ureditev.
\\ \indent Naj bo $X$ končni topološki prostor. Za vsako točko $x \in X$ obstaja najmanjša odprta množica $U_x$, ki jo
vsebuje, oziroma presek vseh odprtih množic, ki vsebujejo $x$. Ta množica je odprta, saj je topologija zaprta za končne preseke.
    Točke uredimo s pravilom $ x\le y \text{, če } U_x \subseteq  U_y$. S tem dobimo šibko ureditev. 
    Antisimetričnost po definiciji sovpada z lastnostjo $T_0$, zato, relacija postane delna ureditev,
     natanko takrat, ko je topologija $T_0$, in disktretna, ko je topologija $T_1$.
    \\ \indent Obratno, naj bo $X$ šibko urejena množica. Na njej lahko definiramo topologijo z bazo $\{y \in X | y\le x\}_{x \in X}$. Če je
$y \le x$, je $y$ vsebovan v vsaki bazni množici, ki vsebuje $x$, torej je $y \in U_x$. Po drugi strani, če je $y\in
U_x$, potem je $y \in \{y \in X | y \le x\}$, torej velja, da je $y \le x$ natanko tedaj ko je $y \in U_x$. Iz tega je razvidno, da so končni prostori in šibke ureditve enaki objekti, gledani z drugačnega stališča.


Delno urejene množice praviloma predstavljamo s Hassejevimi diagrami.

\begin{definicija}
    \textit{Hassejev diagram} delno urejene množice $X$ je usmerjen graf, katerega ogljišča so točke, povezave pa so urejeni pari $(x,y)$, taki, da je  $x<y$ in ne obstaja tak $z$, da bi veljalo $x<z<y$.
\end{definicija}

Povezave $(x,y)$ ne rišemo s puščico iz $x$ v $y$, ampak bomo $x$ in $y$ povezali z ravno črto in $y$ pisali nad $x$. Če je $(x,y)$ povezava v Hassejevem diagramu končne delno urejene množice, rečemo, da $y$ \textit{pokrije} $x$ in pišemo $x\prec y$.

\begin{primer}
    Primer končne topologije in njenega Hassejevega diagrama
\end{primer}

\begin{definicija}
    Element $x$ je \textit{maksimalni element} delno urejene množice $X$, če $\forall y \in X, y\geq x \Rightarrow y = x$.
    $x$ je \textit{maksimum} v $X$, če $\forall y \in X, x\geq y$.
\end{definicija}

Končna delno urejena množica ima maksimum, natanko tedaj, ko ima enoličen maksimalni element. \textit{Minimalni element} in \textit{minimum} definiramo dualno/dualno?.

Elementa $x$ in $y$ sta \textit{primerljiva}, če je $x\leq y$ ali $y\leq x$. \textit{Veriga} v $X$ je podmožica $S\subseteq X$, v kateri je vsak par elementov primerljiv, \textit{antiveriga} v $X$ je podmožica $S\subseteq X$, v kateri ni noben par elementov primerljiv. 

Odprtim množicam v $X$ ustrezajo \textit{"down seti??"}, zaprtim pa \textit{"up seti??"}. Podmnožica $U$
 šibko urejene množice $X$ je down set, če $\forall x\in X, iz y\leq x$, sledi da $y\in U$. Up set definiramo podobno/analogno/dualno??.
Z $F_x$ definiramo zaprtje množice $\{x\}$. $F_x=\{y\in X; y\geq x\}$. Vidimo, da $y\in F_x \Leftrightarrow x\in U_y$.

Tudi morfizmi šibko urejenih množic in morfizmi končnih topoloških prostorov sovpadajo.
  Morfizem šibko urejene množice je preslikava, ki ohranja urejenost torej $f: X\rightarrow Y$, 
  za katero iz $x\leq x'$ sledi $f(x)\leq f(x')$ za vsaka $x,x'\in X$. Morfizmi topoloških prostorov so pa zvezne preslikave.

\begin{trditev}
Funkcija $f:X\rightarrow Y$ med končnima prostoroma je zvezna, natanko tedaj, ko ohranja urejenost.
\end{trditev}

\begin{dokaz}
    Naj bo $f$ zvezna in naj $x\leq x'$ za $x, x' \in X$. Zaradi zveznosti je $f^{-1}(U_{f(x')})$ odprta. Ker velja $f(x')\in U_{f(x')}$, sledi, da $x'\in f^{-1}(U_{f(x')})$, ker je to down set, je tudi $x\in f^{-1}(U_{f(x')})$, na "enakosti" uporabimo $f$ in dobimo $f(x)\in U_{f(x')}$, torej $f(x)\leq f(x')$ in $f$ ohranja urejenost.

    Naj bo zdaj $f$ preslikava, ki ohranja urejenost. Pokažimo, da je $f^{-1}(U_y)$ down set za vsako bazno množico $U_y$. Naj bo $x\leq x'$ in $x'\in f^{-1}(U_y)$, torej $f(x') \in U_y$, ker f ohranja urejenost in je $U_y$ down set, sledi da $f(x)\in U_y$, zato je $x\in f^{-1}(U_y)$, torej je $f^{-1}(U_y)$ down set, torej odprt.


\end{dokaz}


\begin{lema}\label{lem:pot}
    Za vsaki primerljivi točki $x,y\in X$ v končnem prostoru $X$ obstaja pot od $x$ do $y$, tj. preslikava $\alpha: I \rightarrow X$, za katero velja $\alpha(0)=x$ in $\alpha(1)=y$.

\end{lema}
\begin{dokaz}
    Naj bo $x \leq y$. Definirajmo $\alpha:I\rightarrow X$, z $\alpha([0,1))=x$ in $\alpha(1)=y$ in naj bo $U\in X$ odprta. Če je $U$ vsebuje $y$, mora vsebovati tudi $x$, 
    zato je praslika od $U$ ali $\emptyset$ ali $[0,1)$ ali pa $I$, ki so pa vse odprte v $I$, zato je $\alpha$ pot od $x$ do $y$.
\end{dokaz}
Ta lema nam pove, da v končnih prostorih obstajajo netrivialne poti, zato v splošnem fundamentalna grupa končnega prostora ni trivialna.

Naj bosta $X$ in $Y$ končni šibki ureditvi. Z $Y^X$ označimo končno množico zveznih preslikav iz $X$ v $Y$ in jo opremimo z "ureditvijo po točkah" in sicer $f\leq g$, če velja $f(x) \leq g(x), \forall x\in X$. S tem dobimo na $Y^X$ delno ureditev in topologijo. \textit{Ograja} v $X$ je zaporedje $x_0,x_1,...,x_n$ točk v $X$, taka, da sta vsaki zaporedni točki primerljivi. $X$ je \textit{order 
connected}, če za vsaki točki $x,y\in X$ obstaja ograja, ki se začne z $x$ in konča z $y$.
\begin{lema}
    Naj bo $X$ končen prostor. Naslednje trditve so ekvivalentne:

    \begin{itemize}
        \label{lem:povezanost}
        \item $X$ je povezan prostor.
        \item $X$ je order-connected šibka ureditev.
        \item $X$ je povezan s potmi.
    \end{itemize}
\end{lema}


\begin{dokaz}
    Če je $X$ order connected, potem je po lemi \ref{lem:pot}, povezan tudi s potmi.
    Dokazati je treba le še da order-connectedness sledi iz povezanosti. Naj bo torej $X$ povezan, $x\in X$ in $A=\{y\in X| \text{obstaja ograja med $x$ in $y$}\}$. Če 
    je $z\leq x$, potem je tudi $z\in A$, zato je $A$ down set. Analogno pokažemo, da je $A$ up set. Ker je $X$ povezan, sledi, da $A=X$, zato je $X$ order connected.
\end{dokaz}

\begin{trditev}
    \label{iz:ograje}
Naj bosta $f,g: X\rightarrow Y$ preslikavi med končnima prostoroma in $A\subseteq X$, potem je $f\simeq g$ rel $A$, natanko tedaj, ko obstaja ograja $f=f_0\leq f_1\geq ... f_n=g$, taka da $f_i|A=f|A$. Če je $A=\emptyset$, dobimo navadno homotopijo med $f$ in $g$
\end{trditev}

\begin{dokaz}
    Obstoj homotopije $H:f\simeq g$ rel $A$ je ekvivalenten obstoju take poti $\alpha: I \rightarrow Y^X$, da velja $\alpha(t)|A=f|A$, kar je ekvivalentno obstoju poti 
    $\alpha: I \rightarrow M$, kjer je $M\subseteq Y^X$, taka, ki vsebuje preslikave, ki na $A$ sovpadajo z $f$. Po lemi \ref{lem:povezanost} to pomeni, da obstaja ograja 
    med $f$ in $g$ v $M$.
\end{dokaz}

\begin{trditev}
    Naj bo $X$ končen prostor in naj bo $X_0$ kvocient $X/_\sim$, pri čemer $x\sim y \Leftrightarrow x\le y$ in $y\le x$. Potem je $X_0\in T_0$, kvocientna projekcija $q:X\rightarrow X_0$ pa je homotopska ekvivalenca.
\end{trditev}

\begin{dokaz}
    Naj bo $i:X_0\rightarrow X$ katerakoli preslikava, da velja $qi=1_{X_0}$, $i$ ohranja ureditev, zato je zvezna. Ker velja tudi $iq \leq 1_X$, je $i$ homotopski inverz od $q$.

    Naj bosta $x,y\in X$ taka, da $q(x)\leq q(y)$. Po definiciji je $iq \leq 1_X$ in $iq \geq 1_X$, zato je $x \leq iq(x) \leq iq(y) \leq y$. Če velja še $q(y)\leq q(x)$, potem je tudi $y\leq x$, ampak potem je $q(x)=q(y)$, zato je šibka ureditev na $X_0$ antisimetrična, torej je $X_0\in T_0$.
\end{dokaz}


    Ker je $iq\leq 1_X$ ter $iq$ in $1_X$ sovpadata na $X_0$ je po trditevi \ref{iz:ograje} 
    $iq \simeq 1_{X_0}$ rel $X_0$, zato je $X_0$ krepak deformacijski retrakt od $X$.


\begin{definicija}
    Točka $x \in X$ je \textit{navzdol odpravljiva}, če ima $\{y\in X | y \le x\}$ maksimum in \textit{navzgor odpravljiva}, če ima $\{y\in X | y \ge x\}$ minimum. 
    Točka je odpravljiva, če je eno ali drugo.
\end{definicija}

\begin{primer}
    Primer odpravljive točke
\end{primer}

\begin{trditev}
Naj bo $X$ $T_0$ prostor in $x\in X$ odpravljiva točka, potem je $X\backslash \{x\}$ krepak deformacijski retrakt od $X$.
\end{trditev}

\begin{dokaz}
Recimo, da je $x$ navzdol odpravljiva točka, in naj bo $y$ 
maksimum v $U_x$. Definirajmo retrakcijo $r:X\rightarrow 
X\backslash \{x\}$ z $r(x')=x'$ za $x'\neq x$ in $r(x)=y$, 
$r$ ohranja red, saj je $x\leq y$. Če z $i:X\backslash\{x\} 
\rightarrow X$ označimo inkluzijo, je $ir\leq 1_X$, zato je 
po lemi \ref{iz:ograje} je potem $ir \simeq 1_x$ rel 
$X\backslash\{x\}$. Če je $x$ navzgor odpravljiva točka, je 
dokaz analogen.
\end{dokaz}

\begin{definicija}
    $T_0$ prostor je \textit{minimalen}, če nima odpravljivih točk. Krepak deformacijski retrakt, ki je minimalen prostor imenujemo \textit{jedro} končnega prostora $X$.
\end{definicija}

Končnemu prostoru $X$ postopoma odstranjujemo odpravljive točke in s tem v vsakem koraku dobimo prostor, ki je homotopen prostoru $X$, zato je jedro krepak deformacijski retrak začetnega prostora, torej mu je homotopen. Seveda so tudi vsa jedra istega prostora homotopna.

\begin{izrek}
    \label{iz:identiteta}
    Naj bo $X$ končen minimalen prostor. Preslikava $f:X\rightarrow X$ je homotopna identiteti, natanko tedaj, ko je $f=1_X$.
\end{izrek}

\begin{dokaz}
    Po izreku \ref{iz:ograje} lahko predpostavimo, ali 
    $f\leq 1_X$ ali $f\geq 1_X$. %zakaj je to res???
    Pa recimo, da $f\leq 1_X$. 
    Naj bo $x\in X$, trditev dokažimo z indukcijo na 
    število elementov v $U_x$. Če $U_x=\{x\}$, potem je 
    $f(x)=x$, ker $f$ ohranja red, če $U_x\neq\{x\}$, potem 
    je po indukcijski predpostavki 
    $f|_{\hat{U}_x}=1_{\hat{U}_x}$. Če $f(x)=x$, potem je 
    $f(x)\in \hat{U}_x$ in $\forall y < x, y=f(y)\leq 
    f(x)$, torej je $f(x)$ maksimum od $\hat{U}_x$ in je 
    $x$ navzdol odpravljiva točka, kar je pa v protislovju 
    z minimalnostjo prostora $X$. Če je $f\geq 1_X$, je 
    dokaz podoben.
\end{dokaz}

\begin{posledica}
    Homotopska ekvivalenca med minimalnima končnima prostoroma je homeomorfizem. Jedro končnega prostora je enolično do homeomorfizma in dva končna prostora sta homotopna natanko tedaj, ko imata homeomorfna jedra.
\end{posledica}

\begin{dokaz}
    Naj bo $f:X\rightarrow Y$ homotopska ekvivalenca med 
    končnima prostoroma in $g:Y\rightarrow X$ njen inverz. 
    Potem $fg\simeq 1_Y$ in $gf \simeq 1_X$, po trditvi 
    \ref{iz:identiteta} je potem $fg = 1_Y$ in $gf = 1_X$,
    %to je verjetno treba dokazati? 
    torej je $g$ inverz od $f$ in $f$ je homeomorfizem. Če 
    sta $X_0$ in $X_1$ dve jedri končnega prostora $X$, sta 
    sta homotopni, torej med njima obstaja homotopska ekvivalenca $f$, 
    ki je tudi homeomorfizem, torej sta jedri homeomorfni. 
    Prostora $X$ in $Y$ sta istega homotopsko ekvivalentni, 
    natanko tedaj, ko imata homotopsko ekvivalentni jedri, 
    kar pa je tedaj, ko sta jedri homeomorfni.
\end{dokaz}


\begin{trditev}
Naj bo $X$ končen $T_0$ prostor, potem je $X$ minimalen končen prostor, natanko tedaj, ko če $\forall x,y\in X$ velja, da če je  $\forall z\in X$ ki je primerljiv z $x$, primerljiv tudi z $y$, potem sledi da $x=y$
\end{trditev}

\begin{dokaz}
    Najprej negiramo obe strani ekvivalence. predpostavimo, da $X$ ni minimalen, potem obstaja odpravljiva točka $x$. Brez škode za splošnost predpostavimo, da je $x$ navzdol odpravljiva in naj bo $y$ maksimum od $\hat{U}_x$. Če $z\geq x$, potem je $z\geq y$, če pa je $z\le x$, potem je $z\leq y$, ampak $x\neq y$.

    Recimo zdaj, da obstajata $x\neq y$, taka da je vsak 
    element ki je primerljiv z $x$ primerljiv tudi z $y$, 
    torej je tudi $x$ primerljiv z $y$. Predpostavimo 
    $x>y$. Naj bo $A=\{z\in X |  z>x \text{ in za vsak $w\in 
    X$, primerljiv z $z$, $z$ je primerljiv z $y$}\}$. $A$ 
    je neprazna, saj je $x\in A$. Naj bo $x'$ minimalni 
    element v $A$. Pokažimo, da je $x'$ navzdol odpravljiva 
    točka in $y=max(\hat{U}_x)$. Naj bo zdaj $z\<x'$, potem 
    je $z$ primerljiv $y$, saj $x'\in A$. Recimo, da $z>y$ 
    in naj bo $w\in X$. Če $w\geq z$, potem je $w\geq y$, 
    torej $z\in A$, kar je pa v protislovju z minimalnostjo 
    $x'$. Zato $z\leq y$, torej je $y$ maksimum v 
    $\hat{U}_x$.
\end{dokaz}
