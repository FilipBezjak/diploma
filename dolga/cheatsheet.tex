\documentclass[a4paper,12pt]{article}
\usepackage[slovene]{babel}
\usepackage[utf8]{inputenc}
\usepackage[T1]{fontenc}
\usepackage{lmodern}
\usepackage{verbatim}

\usepackage{url}
\usepackage{graphicx}
\usepackage{amsmath}
\usepackage{amsthm}
\usepackage{dsfont}
\usepackage{amssymb}




\title{Minimalni končni modeli prostorov}
\author{Filip Bezjak \\ Mentor: dr. Petar Pavešić}


\setlength\parindent{24pt}


\begin{document}

\begin{itemize}
    \item ureditve, Naj X topološki prostor, Ux, je odprta, uredimo s pravilom. je šibka, postane delna.
    \item obratno, šibko urejena. Definirajmo top z bazo. Če $x \le y$... Če $y\in U_x$. Odprte množice SLIDE
    \item tudi preslikave so si podobne
    \item dokaz.1 f zvezna. $x\le y$ praslika $U_{f(y)}$, ker x is te praslike in $x \in U_y$.
    \item obratno. Praslika $f^{-1}(U_y)$ odprta za vsak y, torej, če vsebuje x, mora vsebovati vse manjše. uporabi $f$
    \item simpleks analog trikotnika, predstavimo kot množico n+1 ogljišč
    \item SLIDE
    \item povej zaprt simpleks, definiraj še metriko na njem
    \item definiraj simplicialni kompleks na tablo, nato še geometrijsko realizacijo
    \item SLIDE
    \item pokaži, kako iz končnega $T_0$ prostora dobimo simplicialni kompleks.
    \item kako zgleda točka v geom realizaciji.
    \item pomembni vlogo igra preslikava McCordova SLIDE
    \item Nekaj definicij iz Algebraične topologije. Poti, homotopija poti.
    \item Homotopija je družina preslikav
    \item SLIDE
    \item dokaži
    \item Definirajmo množenje
    \item omejimo se na zanke, množico označimo s $\pi_1$
    \item SLIDE
    \item dokaži. Grupi pravimo fundamentalna grupa.
    \item posploši množenje za poljuben n. posploši še grupo
    \item SLIDE
    \item šibka homotopska ekvivalenca.
    \item SLIDE
    \item model za sfero.
\end{itemize}
\end{document}