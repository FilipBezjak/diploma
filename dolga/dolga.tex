\documentclass[t]{beamer} % [t] pomeni poravnavo na vrh slida

% \usepackage{etex} % vključi ta paket, če ti javi napako, da imaš naloženih preveč paketov.

% standardni paketi
\usepackage[slovene]{babel}
\usepackage[T1]{fontenc}
\usepackage[utf8]{inputenc}
\usepackage{amssymb}
\usepackage{amsmath}
\usepackage{url}
\usepackage{graphicx}

% paket, ki ga rabimo za risanje
\usepackage{tikz}

\usepackage{lmodern}                            % to get rid of font warnings
\renewcommand\textbullet{\ensuremath{\bullet}}  % to get rid of font warnings

% podatki
\title{Minimalni modeli prostorov}
\author{Filip Bezjak}
\institute{mentor: prof. dr. Petar Pavešić}
% tvoj izbran stil predstavitve
\usetheme{Singapore}
\usecolortheme{crane}

% \setbeamertemplate{footline}{}            % pri nekaterih temah je potrebno odstraniti
% \setbeamertemplate{navigation symbols}{}  % nogo, v tem primeru to odkomentiraj

%  ukaz za počrnitev enega slida. Nariše črn pravokotnik višine #1 na dno.
\newcommand{\fillblack}[1]{
\begin{tikzpicture}[remember picture, overlay]
    \node [shift={(0 cm,0cm)}]  at (current page.south west)
        {%
        \begin{tikzpicture}[remember picture, overlay] at (current page.south west)
            \draw [fill=black] (0, 0) -- (0,#1 \paperheight) --
                              (\paperwidth,#1 \paperheight) -- (\paperwidth,0) -- cycle ;
        \end{tikzpicture}
        };
        \draw (current page.north west) rectangle (current page.south east);
\end{tikzpicture}
}

% Projektor, ki sveti na tablo je zoprn, ker zavzema prostor za pisanje.
% Praktična rešitev je, da se platno spusti samo do table, spodnji del
% predstavitve pa naredimo popolnoma črn. Tako projektor na tablo ne projecira
% svetlobe in je pisanje nemoteno s strani predstavitve, ki je cela nad tablo.

% Za to da dobite spodaj zatemnjen slide, je potrebno na konec frame-a dodati
% ukaz \fillblack{delež}, kjer je delež številka med 0 in 1, ki pove, kakšen
% delež slida želite imeti zapolnjen s črnim pravokotnikom. Spodaj so 4 slidi,
% ki imajo začrnjenih: 33%, 50%, 33% in 20%.  Za šolske table v 2.02 in 2.03 je
% 33% ravno prav, da lahko pišete po celi tabli.

% Črni pravokotnik ne zavzema prostora na slidu -- preprosto nariše se čez
% spodnjo tretjino (ali kolikor pač želite). To pomeni, da se pokrijve tudi ves
% tekst spodaj. Za to je na začetku dokumenta tudi nastavljena vertikalna
% poravnava na ``zgoraj'', z razliko od običajne sredinske (to je tisti [t]).
% Tako se ves tekst obdrži na vidnem delu slida, dokler je to možno. Tretji
% slide spodaj demonstrira, kaj se zgodi v primeru preveč besedila.

% Kod demonstrira zadnji slide, je vseeno, kjer napišemo ukaz, toda ponavadi ga
% napišemo na dnu, da se izognemo morebitnim nevšečnostim ali neželenemu
% spacingu. Pri nekaterih temah se izriše tudi noga, ki se izriše na koncu
% slida, torej po tem ko je naš pravokotnik že narisan, kar zna biti moteče.
% To se na primer zgodi tudi pri standardni temi Warsaw. Rešitev je, da nogo
% odstranimo (saj bi bila itak prekrita). V preambuli so zakomentirani ukazi,
% kako to naredimo.

\begin{document}

\begin{frame}
  \maketitle
  \fillblack{0.33}
\end{frame}

\begin{frame}
    \fillblack{0.33}
    \begin{block}{Definicija}
        \textit{šibka ureditev} - tranzitivna in refleksivna relacija. \\
        Če je relacija še antisimetrična, dobimo \textit{delno ureditev}.
    \end{block}
    \pause
    \begin{block}{primer}
        $X = \{a,b,c,d\}$ s topologijo 
        $\{\emptyset,\{a\}, \{b\}, \{a,b,c\}, \{a,b,d\},  X\}$
    \end{block}
    \pause
    \begin{block}{Izrek}
        Preslikava $f: X \longrightarrow Y$ med končnima prostoroma je zvezna, natanko tedaj ko je monotona
    \end{block}
\end{frame}

\begin{frame}
    \fillblack{0.5}
    \begin{block}{definicija}
    \textit{Simpleks} ali $n$-simpleks je $n$-razsežni analog trikotnika.\\
    Zaprt simpleks $\bar{\sigma}$ je množica formalnih konveksnih combinacij $\Sigma_{i=0}^{n}\alpha_i v_i$, pri čemer $\alpha_i \ge 0$ in $\Sigma \alpha_i = 1$.
    \end{block}
    \pause
    \begin{block}{Primer}
        $K=\{\{a\},\{b\},\{c\},\{d\},\{a, b\},\{a, c
   \}, \{ b, c\},\{a, d\}, \{ a, b, c\}\}$
    \end{block}
  \end{frame}

\begin{frame}
    \fillblack{0.5}
    \begin{block}{izrek}
        relacija homotopije na poteh s fiksnima krajiščema je ekvivalenčna relacija za vsak topološki prostor.
    \end{block}
    \pause
    \begin{block}{izrek}
        $\pi_1(X,x_0)$ opremljena s produktom $[f][g] = [f\cdot g]$ je grupa.
    \end{block}
\end{frame}

\begin{frame}
    \fillblack{0.33}
    \begin{block}{izrek}
        $\pi_n(X,x_0)$ opremljena s produktom $[f][g] = [f\cdot g]$ je grupa za vsak $n \in N$.
    \end{block}
    \pause
\begin{block}{definicija}
    Topološka prostora sta \textit{šibko homotopsko ekvivalentna}, če so njune homotopske grupe izomorfne za vsak $n \in N$.    
\end{block}
\end{frame}


  \begin{frame}
    \fillblack{0.5}
    \begin{block}{definicija}
      $\mathcal{K}$-\textit{McCordova} preslikava: $\mu_X:|\mathcal{K}
      (X)|\rightarrow X$, $\mu_X(\alpha) =
      min(\textit{support}(\alpha))$.
    \end{block}
    \pause
    \begin{block}{izrek}
      $\mathcal{K}$-\textit{McCordova} preslikava je šibka homotopska 
      ekvivalenca za vsak končen $T_0$-prostor.
    \end{block}
    \end{frame}

    \begin{frame}
        \fillblack{0.5}
        \begin{block}{definicija}
            Končni topološki prostor je \textit{model} prostora $X$, če mu je šibko homotopsko ekvivalenten. 
      \\  \pause
            Model je \textit{minimalen}, če ima izmed vseh modelov najmanjšo kardinalnost.
        \end{block}
        \end{frame}
    \end{document}
