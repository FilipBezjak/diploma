\documentclass[a4paper,12pt]{article}
\usepackage[slovene]{babel}
\usepackage[utf8]{inputenc}
\usepackage[T1]{fontenc}
\usepackage{lmodern}
\usepackage{verbatim}

\usepackage{url}
\usepackage{graphicx}
\usepackage{amsmath}
\usepackage{amsthm}
\usepackage{dsfont}
\usepackage{amssymb}



\makeatletter
\DeclareRobustCommand{\sqcdot}{\mathbin{\mathpalette\morphic@sqcdot\relax}}
\newcommand{\morphic@sqcdot}[2]{%
\sbox\z@{$\m@th#1\centerdot$}%
\ht\z@=.33333\ht\z@
\vcenter{\box\z@}%
}
\makeatother

\newcommand\mymathop[1]{\mathop{\operatorname{#1}}}

\title{Minimalni končni modeli prostorov}
\author{Filip Bezjak \\ Mentor: dr. Petar Pavešić}


\setlength\parindent{24pt}

\theoremstyle{definition}
\newtheorem{definicija}{Definicija}

\theoremstyle{plain}
\newtheorem{izrek}{Izrek}

\theoremstyle{plain}
\newtheorem{lema}{Iema}

\newenvironment{dokaz}{\begin{proof}[\bfseries\upshape\proofname]}{\end{proof}}

\begin{document}

\maketitle

\section{Uvod}
Moja tema sodi na področje algebraične topologije na končnih prostorih. Topologije na končnih prostorih so večkrat spregledane, saj je vsaka $T_1$ topologija
na končnem prostoru diskretna. Če pa lastnosti $T_1$ ne zahtevamo, postanejo veliko bolj
zanimive.

\section{Končni topološki prostori in delno urejene množice}

\textit{Končni topološki prostor} je topološki prostor s končno mnogo točkami, \textit{šibko urejena} množica je množica s tranzitivno in z refleksivno relacijo. Če je relacija še antisimetrična, dobimo \textit{delno} ureditev.
\\ \indent Naj bo $X$ končni topološki prostor. Za vsako točko $x \in X$ obstaja najmanjša odprta množica $U_x$, ki jo
vsebuje, oziroma presek vseh odprtih množic, ki vsebujejo $x$. Ta množica je odprta, saj je topologija zaprta za končne preseke.
    Točke uredimo s pravilom $ x\le y \text{, če } U_x \subseteq  U_y$. S tem dobimo šibko ureditev. Relacija postane delna ureditev, natanko takrat, ko je topologija $t_0$, in disktretna, ko je topologija $T_1$.
    \\ \indent Obratno, naj bo $X$ šibko urejena množica. Na njej lahko definiramo topologijo z bazo $\{y \in X | y\le x\}_{x \in X}$. Če je
$y \le x$, je $y$ vsebovan v vsaki bazni množici, ki vsebuje $x$, torej je $y \in U_x$. Po drugi strani, če je $y\in
U_x$, potem je $y \in \{y \in X | y \le x\}$, torej velja, da je $y \le x$ natanko tedaj ko je $y \in U_x$. Iz tega je razvidno, da so končni prostori in šibke ureditve enaki objekti, gledani z drugačnega stališča.

\begin{definicija}
    Točka $x \in X$ je \textit{navzdol odpravljiva}, če ima $\{y\in X | y \le x}\}$ maksimum in \textit{navzgor odpravljiva}, če ima $\{y\in X | y \ge x}\}$ minimum. Točka je odpravljiva, če je eno ali drugo.
\end{definicija}

\begin{definicija}
    $T_0$ prostor je \textit{minimalen}, če nima odpravljivih točk.
\end{definicija}

\section{Homotopska in šibka homotopska ekvivalenca}

Pot v prostoru $X$ je zvezna preslikava $ f: I \rightarrow X$, pri čemer je $I$ enotski interval $[0,1]$. Poti sta si homotopni, če lahko eno zvezno deformiramo v drugo, brez da bi premaknili krajišči poti.
\begin{definicija}
    
    Homotopija poti v $X$ je družina preslikav $f_t:I \rightarrow X, 0\le t \le 1$, taka da
    \begin{itemize}
        \item 
        sta krajišči $f_t(0) = x_0$ in $f_t(1) = x_1$ neodvisni od $t$ in
        
        \item 
        je prirejena preslikava $F:I\times X \rightarrow X$ definirana s $F(s,t) = f_t(s)$ zvezna.
    \end{itemize}
    Za preslikavi $f_1$ in $f_0$, ki sta povezani s homotopiijo $f_t$ rečemo, da sta homotopni in označimo $f_1 \simeq f_0$.
\end{definicija}

\begin{definicija}
    Preslikava  $f : X \rightarrow Y$ je \textit{homotopska ekvivalenca} prostorov $X$ in $Y$, če obstaja preslikava $g: Y\rightarrow X$, taka da
    je $f g \simeq \mathds{1}$ in $gf \simeq \mathds{1}$. Rečemo, da sta si prostora $X$ in
    $Y$  \textit{homotopsko ekvivalentna}.
\end{definicija}

\begin{izrek}
    relacija homotopije na poteh s fiksnima krajiščema je ekvivalenčna relacija za vsak topološki prostor.
\end{izrek}

Za poljubni poti $f,g : I \rightarrow X$, za kateri velja $f(1) = g(0)$ lahko definiramo produkt $f\sqcdot g$, ki preteče f in g z dvojno hitrostjo v enotskem intervalu.
$$ f\sqcdot g(s) \begin{cases}
    f(2s), & 0\le s\le 1/2 \\
    g(2s-1), & 1/2 \le s \le 1
\end{cases}
$$

Če se omejimo samo na poti $f:I \rightarrow X$ z enako začetno in končno točko $f(0) = f(1) = x_0$, govorimo o zankah, za $x_0$ pa rečemo, da je bazna točka.
Množico vseh homotopskih razredov $[f]$, z bazno točko $x_0$ označimo z $\pi_1(X,x_0)$.

\begin{izrek}
    $\pi_1(X,x_0)$ opremljena s produktom $[f][g] = [f\sqcdot g]$ je grupa.
\end{izrek}

Tej grupi pravimo fundamentalna grupa prostora $X$, z bazno točko $x_0$. $\pi_1(X,x_0)$ je prva v zaporedju analogogno definiranih grup $\pi_n(X,x_0)$, pri katerih namesto iz $I$ slikamo iz $n$-dimenzionalne kocke $I^n$.


Naj bo $I^n$ $n$-dimenzionalna kocka. Rob $\partial I^n \text{ od } I^n$ je podprostor točk pri katerih je vsaj ena koordinata enaka $1$ ali $0$. Definirajmo $\pi_n(X,x_0)$, množico homotopskih razredov preslikav $f:(I^n,\partial I^n) \rightarrow (X,x_0)$ pri čemer velja $f(\partial I^n) = x_0$.


Za $n\ge 2$ posplošimo množenje definirano pri fundamentalni grupi.
$$ f\sqcdot g(s) \begin{cases}
    f(2s_1,s_2,\ldots,s_n), & 0\le s_1\le 1/2 \\
    g(2s_1-1,s_2,\ldots,s_n), & 1/2 \le s_1 \le 1
\end{cases}
$$

\begin{izrek}
    $\pi_n(X,x_0)$ opremljena s produktom $[f][g] = [f\sqcdot g]$ je grupa za vsak $n \in \mathds{N}$.
\end{izrek}

Grupam $\pi_n(X,x_0)$ pravimo \textit{homotopske grupe}.

\begin{definicija}
    Topološka prostora sta \textit{šibko homotopsko ekvivalentna}, če so njune homotopske grupe izomorfne za vsak $n \in \mathds{N}$.
\end{definicija}

Homotopsko ekvivalentni prostori so si tudi šibsko homotopsko ekvivalentni.

\begin{definicija}
    Preslikava je \textit{šibka homotopska ekvivalenca}, če preko kompozicije inducira izomorfizem na vse homotopske grupe.
\end{definicija}


\section{Simpleksi}

\textit{Simpleks} ali $n$-simpleks je $n$-razsežni analog trikotnika. Točka je $0$-simpleks, $1$-simpleks je daljica, $2$-simpleks je trikotnik,
$3$-simpleks je tetraeder. $n$-simpleks definiramo kot množico svojih $n+1$ oglišč.

\textit{Simplicialni kompleks $K$} je sestavljen iz množice oglišč $V_K$ in množice simpleksov $S_K$, sestavljene iz končnih nepraznih podmnožic od $V_k$, pri čemer je vsak element $S_k$ simpleks in vsaka podmnožica simpleksa je simpleks. 


Naj bo $\sigma = \{v_0,v_1,\ldots,v_n\}$ $n$-simpleks. Zaprt
Simpleks $\bar{\sigma}$ je množica formalnih konveksnih combinacij $\mymathop{\Sigma}_{i=0}^{n}\alpha_i v_i$
pri čemer je $\alpha_i \ge 0$ za vsak $0\le i \le n$ in $\Sigma \alpha_i = 1$. Zaprt simpleks je metričen prostor z metriko
$$
d(\underset{v \in K}{\Sigma}\alpha_v v,\underset{v \in K}{\Sigma}\beta_v v) = \sqrt{\underset{v \in K}{\Sigma}(\alpha_v - \beta_v)^2}
$$

\textit{Geometrijska realizacija} $|K|$ simplicialnega kompleksa $K$ je 
množica formalnih konveksnih kombinacij $\underset{v \in K}{\Sigma}\alpha_v v$, takih da je $\{v | \alpha_v \textgreater 0\}$ simpleks v $K$.


\section{Minimalni modeli prostorov}

\begin{definicija}
    Končni topološki prostor je \textit{model} prostora $X$, če mu je šibko homotopsko ekvivalenten. Model je \textit{minimalen}, če ima izmed vseh modelov najmanjšo kardinalnost.
\end{definicija}

\begin{izrek}
    Homotopska ekvivalenca med minimalnima $T_0$ prostoroma je homeomorfizem.
\end{izrek}

\begin{izrek}Naj bo $X$ $T_0$ prostor in $x$ navzdol odpravljiva točka, tedaj je $r: X \rightarrow X - \{x\}$, $$
    r(u) = \begin{cases}
        u, & u \neq x \\
        max(u), & u = x
    \end{cases}$$
homotopska ekvivalenca.
\end{izrek}

Preslikavo lahko analogno definiramo za navzgor odpravljive točke, le da namesto v \textit{max(u)} slikamo v \textit{min(u)}. 
Iz poljubnega modela prostora torej dobimo minimalnega, s postopnim odstranjevanjem odpravljivih točk.

\begin{definicija}
    Vsak končen $T_0$-prostor $X$ ima \textit{prirejen} simplicialni kompleks $\mathcal{K}(X)$, katerega simpleksi so neprazne verige v prirejeni delni urejenosti na $X$.
\end{definicija}



Točka $\alpha$ v geometrijski realizaciji $|\mathcal{K(X)}|$ je
konveksna kombinacija oblike
$\alpha = t_1x_1+t_2x_2 + \ldots + t_r x_r$, pri čemer 
$\sum_{i=1}^{r}t_i=1$, za vsak $1 \le i \le r$, $t_i \ge 0$ in 
velja, da je $x_1 \textless x_2 \textless \ldots \textless x_r$ veriga v $X$.
 Nosilec $\alpha$ je množica \textit{support}($\alpha$)$= \{x_1,x_2,\ldots,x_r\}$. Pomembno vlogo igra 
 preslikava $\alpha \mapsto x_1$.
 
 \begin{definicija}
    Naj bo $X$ končen $T_0$ prostor, Definirajmo
    $\mathcal{K}$-\textit{McCordovo} preslikavo $\mu_X:|\mathcal{K}
    (X)|\rightarrow X$, z $\mu_X(\alpha) =$
    min(\textit{support}($\alpha))$.
\end{definicija}

\begin{izrek}
    $\mathcal{K}$-\textit{McCordova} preslikava je šibka homotopska 
    ekvivalenca za vsak končen $T_0$-prostor.
\end{izrek}


Moja naloga v tem delu bo poiskati minimalne končne modele sfer in grafov.


\nocite{*}


\bibliography{osnutek}
\bibliographystyle{plain}

\end{document}